\documentclass[11pt]{article}

% This is not final, I'm just filling in a title
\title{Research Proposal}
\author{Fabio Molo and Jonas Arnfred}

\usepackage{natbib}
\usepackage{hyperref}
\begin{document}

\maketitle

\section{Context}

Chinese share in world trade has increased four times over the last decade. This increase has been mainly driven by increased 
exports which by now far surpass China's imports.
China has been accused of artificially aiding these exports by 
manipulating the exchange rate of the Renminbi (RMB). By keeping the value of 
the Renminbi low, it becomes cheaper for western consumers to buy products produced in China 
, and more difficult for western producers to sell western products on the Chinese market.
This accusation as become one of the major diplomatic issues between the 
United States and China during the last ten years. In the political 
discourse US politicians are accusing China of stealing American jobs on 
unfair terms, while the Chinese administration claims that the RMB is 
not artificially undervalued.  Each use economic arguments to defend 
their claim, and this usage of economic theory in diplomatic discourse 
shed light on the practical value of the field of economics in economic 
diplomacy.

\section{Research question}

We intend to illuminate this issue by first exploring how the economic 
arguments for both sides of the debate are presented in academic discourse. We will not enter the academic economic debate but we will try to convey the economic 
arguments as clearly and simple as possible, meant for a reader without any 
background in economic theory.  

With the economic argumentation in mind we explore the resulting 
political discourse and analyse how this discourse is connected with the 
underlying economic theory.  Based on these observations we will try to 
judge the validity of the stances both for and against the RMB and 
discuss how the economic theories influence the decisions. Are the 
politicians well informed in their argumentation? Do economic arguments prove decisive or is 
economics used purely as guise for a political agenda?

\section{Method}

By analyzing academic literature arguing for and against an artificial 
depreciation of the RMB and holding these arguments up against 
statements made in official speeches by politicians we hope to make 
clear how the economic theory corresponds with diplomatic practice.

\section{Structure}

\begin{itemize}
\item{Introduction: We contextualize the theme and frame the research 
question, and outline the methods used in the report.}

\item{Economics 101: What are the basic models and theories that come in 
to play when we talk about undervaluation of the RMB?}

%\blankline
\item{Applied economics: Within this basic framework we will flesh out 
the arguments made for and against the appreciation of the RMB in the 
academic discourse.}

%\blankline
% In between we have people like Krugman who spearheads the political 
% discourse while being an economist
\item{The political debate: China has been heavy criticized by the 
United States during the last decade. We would like to highlight the 
arguments used by both sides in the diplomatic debate, with a special 
attention on who is saying what.}

%\blankline
\item{Discussion of economic arguments: Based on the academic discussion 
on the issue we intend to discuss the merit of the different viewpoints 
and what part of economics are used by either side.}

%- Discussion of political arguments: 
%
%\blankline
\item{Discussion of political use of economics: We intend to discuss how 
politicians are making use of economic arguments and try to use 
knowledge of economics and common sense to judge if they have any root 
in reality.}

\item{Conclusion}

\end{itemize}

\nocite{*}
\bibliographystyle{abbrv}
\bibliography{research_proposal}

\end{document}
