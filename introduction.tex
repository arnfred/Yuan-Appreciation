\section{Introduction}
\label{sec:introduction}

In the presidential election of 2012 the Republican candidate Mitt 
Romney made a tough stance against the currency policy of the People's 
Bank of China: ``Unless China changes its ways'' he stated, `` on day 
one of my presidency I will designate it a currency manipulator and take 
appropriate 
counteraction.''\footnote{http://www.guardian.co.uk/world/video/2012/oct/23/debate-obama-romney-china-video}.

The consequences of globalization in terms of off-shoring and loss of 
manufacturing jobs have been a recurrent issue in American politics over 
the last decades. Still, from an outside perspective it is a 
curious fact that a seemingly technocratic and complicated issue like 
China's currency policy is given such an important role in US politics.  

In this paper we try to answer the question if the Chinese currency 
policy deserves such a central position in the United States' foreign 
policy debate given its actual economic importance and the United 
States' possibilities to influence it.

% Jonas: TODO maybe add something like "In particular we will try to 
% answer the following questions: yadda yadda yadda

In the chapter ``\nameref{sec:economics}'', we will explore the 
underlying economic theory, focusing on standard textbook economic 
models.  What is the importance of the exchange rate?  What does it mean 
for a currency to be undervalued? And how do economists know if and by 
how much a currency is undervalued? We will then portray the academic 
debate among economists about China's currency policy. 

In the following chapter ``\nameref{sec:politics}'', we will look at the 
political debate: How has it developed in the US over the last ten 
years?  What are the arguments brought forward and who are the main 
participants? On the Chinese side, what are the political arguments for 
(and against) current Chinese monetary policy?  

In the third chapter ``\nameref{sec:diplomacy}'' we will discuss the 
issue from a diplomatic point of view: What are the diplomatic options 
that the USA have to put pressure on China to change its foreign 
exchange policy? 

In the final chapter ``\nameref{sec:discussion}'' we will bring together 
the economic, political and diplomatic threads to answer the question 
how economics inform the diplomatic and political debate about China's 
foreign exchange policy. 
