\section{Introduction}

In the presidential election 2012 the Republican candidate Mitt Romney made a tough stance against the currency policy of the People's Bank of China: In the presidential debate on foreign policy of October 23, he declared that `on day one in office, I will declare China a currency manipulator'.\footnote{http://www.guardian.co.uk/world/video/2012/oct/23/debate-obama-romney-china-video}%TODO: check the exact wording.
Already in 2008, then presidential candidate Barack Obama made a similar pledge.\footnote{\cite{Obama2008}}

The Chinese reply on a equally high level. Premier Wen Jiabao

How comes a seemingly technocratic and complicated issue like China's currency policy plays such an important role in US politics? And, given its actual economic importance and the United States' possibilities to influence it, does the Chinese currency policy deserve such a central position in the United States' foreign policy debate? %and something about China?

In the first Chapter, we will look at the economics of the issue. Why could a undervalued currency be important? What does it mean for a currency to be undervalued? And how do economists know if and by how much a currency is undervalued? We will focus on standard, textbook economic models and academic debates among economists about China's currency policy. 

In the next Chapter, we will look at the political debate: How has it developped in the US over the last ten years? What are the arguments brought forward and who are the main participants? On the Chinese side, what are the political arguments for (and against) current Chinese monetary policy? In this chapter, officials' statements and press releases are the main source of analysis. %TODO: It would be nice to find more press material.

%fmolo: Maybe there should be a section "diplomacy"?
In the third chapter we will discuss the issue from a diplomatic point of view: What are the diplomatic options for the USA to put pressure on China to change its foreign exchange policy? %and something about China? 

In the final chapter we will bring together the economic, political and diplomatic threads to answer the question how economics inform the diplomatic and political debate about China's foreign exchange policy.



%fmolo: possibly move part of the following paragraphs in the economics section.

What makes Chinese goods competitive on the world market? One might be 
tempted to point out the hard work, innovation and creativity of the 
Chinese working force.\footnote{As \cite[p. 18]{Yu2010} indirctly does.} 
Not being convinced by this, economists have brought forward several 
other, more structural explanations:

One factor is \emph{labor arbitrage}:\footnote{This factor was pointed at 
by Xu Mingqi of the Institute of World economy of the Shanghai Academy 
of Social Sciences in a talk to our class on September 4 2012.} Chinese 
workers are willing to work at lower wages than workers in importing 
countries. Importantly, accepted wages are not only lower in absolute 
terms but also in terms of purchasing power: A typical wage in China 
allows for a lower standard of living than a typical wage in an 
industrial country, thereby allowing Chinese firms to produce with much 
lower labor costs, both absolute and relative. 

Additionally Chinese producers rely cheap energy and land 
rents.\footnote{\cite[pp.  25]{Huang2010}.} These markets are not 
liberalized and prices can therefore be strongly influenced by 
government policy. For Chinese officials on the local as well as on the 
federal level GDP growth is a major ambition, they maintain energy and 
land use prices that are cheaper on average than in industrial countries 
as well as other emerging economies.


% Jonas: Merge with section in politics
% Another factor explaining strong Chinese exports has been introduced 
% in 2005 by Ben Bernanke, shortly before he was named chairman of the 
% US Federal Reserve.\footnote{\cite{Bernanke2005}}: The \emph{saving 
% glut hypothesis}. According to Bernanke a special series of 
% circumstances has lead to exceptionally high saving rate, i.e. the 
% percentage of income that is saved. These circumstances include 
% repercussions of the financial crises in emerging economies in the 
% late 90's, but also the unique saving behaviour of Chinese households. 
% Partly due to the lack of social security institutions and to the 
% One-Child-Policy, the saving rate of Chinese households is among the 
% highest in the world - in 2007 it was 53\% as opposed to Switzerland's 
% 17,5\%.\footnote{\cite[pp. 20]{Taoyang2011}.}\footnote{Swiss Federal 
% Statistics Office, 
% http://www.bfs.admin.ch/bfs/portal/en/index/themen/00/09/blank/ind42.indicator.420004.420001.html.} 
% These savings drive down interest rates in China and allow the local 
% producers to access very cheap loans, which in turn allows them to 
% expand production.\footnote{This explanation is also favoured by 
% \cite[pp. 41]{Wyplosz2010}.}

An alternative explanation that has been proposed, especially by 
politicians and economists in the US, is that China conducts an 
\emph{exchange rate policy} which makes Chinese goods more competitive.
Their argument is that China is maintining the Renminbi (RMB) at an 
artificially low value making it cheaper for foreign nations to buy 
chinese goods and more expensive for the Chinese to buy foreign goods.  
During the last decade it has been commonplace to hear American 
politicians argue the unfairness of this practice, creating a heated 
diplomatic debate between these two countries.

In this article we intend to explore this issue by looking at it from 
different angles. To understand the mechanics of China's policies and 
the tools that we can use to judge the fairness of them, we look at the 
discussion as it has taken place in academic litteratue. Armed with this 
knowledge we turn to the politicians and look at how they present their 
arguments and for what reasons. Finally we aim to look at the academic 
and diplomatic debate in order to answer two questions:

% Jonas: These questions can be changed to fit whatever we end up 
% writing
\begin{itemize}
\item{How does the political debate relate to the underlying economic 
principles and what explains the differences in viewpoints?}
\item{With what right can the United States expect China to change their 
economic policy based on their argumentation?}
\end{itemize}
