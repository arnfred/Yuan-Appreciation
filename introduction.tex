\section{Introduction}

What makes Chinese goods competitive on the world market? One might be 
tempted to point out the hard work, innovation and creativity of the 
Chinese working force.\footnote{As \cite[p. 18]{Yu2010} indirctly does.} 
Not being convinced by this, economists have brought forward several 
other, more structural explanations:

One factor is \emph{labor arbitrage}:\footnote{This factor was hinted at 
by Xu Mingqi of the Institute of World economy of the Shanghai Academy 
of Social Sciences in a talk to our class on September 4 2012.} Chinese 
workers are willing to work at lower wages than workers in importing 
countries. Importantly, accepted wages are not only lower in absolute 
terms but also in terms of purchasing power: A typical wage in China 
allows for a lower standard of living than a typical wage in an 
industrial country, thereby allowing Chinese firms to produce with much 
lower labor costs, both absolute and relative. 

Additionally Chinese producers rely cheap energy and land 
rents.\footnote{\cite[pp.  25]{Huang2010}.} These markets are not 
liberalized and prices can therefore be strongly influenced by 
government policy. For Chinese officials on the local as well as on the 
federal level GDP growth is a major ambition, they maintain energy and 
land use prices that are cheaper on average than in industrial countries 
as well as other emerging economies.


% Jonas: Merge with section in politics
% Another factor explaining strong Chinese exports has been introduced 
% in 2005 by Ben Bernanke, shortly before he was named chairman of the 
% US Federal Reserve.\footnote{\cite{Bernanke2005}}: The \emph{saving 
% glut hypothesis}. According to Bernanke a special series of 
% circumstances has lead to exceptionally high saving rate, i.e. the 
% percentage of income that is saved. These circumstances include 
% repercussions of the financial crises in emerging economies in the 
% late 90's, but also the unique saving behaviour of Chinese households. 
% Partly due to the lack of social security institutions and to the 
% One-Child-Policy, the saving rate of Chinese households is among the 
% highest in the world - in 2007 it was 53\% as opposed to Switzerland's 
% 17,5\%.\footnote{\cite[pp. 20]{Taoyang2011}.}\footnote{Swiss Federal 
% Statistics Office, 
% http://www.bfs.admin.ch/bfs/portal/en/index/themen/00/09/blank/ind42.indicator.420004.420001.html.} 
% These savings drive down interest rates in China and allow the local 
% producers to access very cheap loans, which in turn allows them to 
% expand production.\footnote{This explanation is also favoured by 
% \cite[pp. 41]{Wyplosz2010}.}

An alternative explanation that has been proposed, especially by 
politicians and economists in the US, is that China conducts an 
\emph{exchange rate policy} which makes Chinese goods more competitive.
Their argument is that China is maintining the Renminbi (RMB) at an 
artificially low value making it cheaper for foreign nations to buy 
chinese goods and more expensive for the Chinese to buy foreign goods.  
During the last decade it has been commonplace to hear American 
politicians argue the unfairness of this practice, creating a heated 
diplomatic debate between these two countries.

In this article we intend to explore this issue by looking at it from 
different angles. To understand the mechanics of China's policies and 
the tools that we can use to judge the fairness of them, we look at the 
discussion as it has taken place in academic litteratue. Armed with this 
knowledge we turn to the politicians and look at how they present their 
arguments and for what reasons. Finally we aim to look at the academic 
and diplomatic debate in order to answer two questions:

% Jonas: These questions can be changed to fit whatever we end up 
% writing
\begin{enumeration}
\item{How does the political debate relate to the underlying economic 
principles and what explains the differences in viewpoints?}
\item{With what right can the United States expect China to change their 
economic policy based on their argumentation?}
\end{enumeration}
