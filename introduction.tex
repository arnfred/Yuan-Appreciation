\section{Introduction}

In the presidential election 2012 the Republican candidate Mitt Romney made a tough stance against the currency policy of the People's Bank of China: In the presidential debate on foreign policy of October 23, he declared that `on day one in office, I will declare China a currency manipulator'.\footnote{http://www.guardian.co.uk/world/video/2012/oct/23/debate-obama-romney-china-video}%TODO: check the exact wording.
Already in 2008, then presidential candidate Barack Obama made a similar pledge.\footnote{\cite{Obama2008}}

The Chinese reply on a equally high level. Premier Wen Jiabao

How comes a seemingly technocratic and complicated issue like China's currency policy plays such an important role in US politics? And, given its actual economic importance and the United States' possibilities to influence it, does the Chinese currency policy deserve such a central position in the United States' foreign policy debate? %and something about China?

In the first Chapter, we will look at the economics of the issue, first focusing on standard, textbook economic models. What is the importance of the exchange rate? What does it mean for a currency to be undervalued? And how do economists know if and by how much a currency is undervalued? We will then portray the academic debate among economists about China's currency policy. 

In the next Chapter, we will look at the political debate: How has it developped in the US over the last ten years? What are the arguments brought forward and who are the main participants? On the Chinese side, what are the political arguments for (and against) current Chinese monetary policy? In this chapter, officials' statements and press releases are the main source of analysis. %TODO: It would be nice to find more press material.

%fmolo: Maybe there should be a section "diplomacy"?
In the third chapter we will discuss the issue from a diplomatic point of view: What are the diplomatic options for the USA to put pressure on China to change its foreign exchange policy? %and something about China? 

In the final chapter we will bring together the economic, political and diplomatic threads to answer the question how economics inform the diplomatic and political debate about China's foreign exchange policy.



%fmolo: possibly move part of the following paragraphs in the economics section.

%In this article we intend to explore this issue by looking at it from 
%different angles. To understand the mechanics of China's policies and 
%the tools that we can use to judge the fairness of them, we look at the 
%discussion as it has taken place in academic litteratue. Armed with this 
%knowledge we turn to the politicians and look at how they present their 
%arguments and for what reasons. Finally we aim to look at the academic 
%and diplomatic debate in order to answer two questions:

% Jonas: These questions can be changed to fit whatever we end up 
% writing
%\begin{itemize}
%\item{How does the political debate relate to the underlying economic 
%principles and what explains the differences in viewpoints?}
%\item{With what right can the United States expect China to change their 
%economic policy based on their argumentation?}
%\end{itemize}
