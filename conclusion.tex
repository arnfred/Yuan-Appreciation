
\section{Conclusion}

We have layed out the basics of the economics, the politics and the economic diplomacy of the debate on RMB undevaluation

Regarding the economics we find it clear that China is indeed trying to keep a low currency. Chinese monetary policy takes every step that is suited to do so. Pointing out other possible explanations for China's export success is not sufficient to show that RMB undervaluation plays no role whatsoever. Even though it is hard to quantify the effect of the low RMB, we don't know how one could have a more clear case of a undervalued currency.

At the same time the political debate in the USA seems hyperbolic to us. Havin a currency valued below the equilibrium exchange rate is not the same as `manipulating' a currency or even `unfairly' doing so. Every country is using monetary policy to satisfy its domestic goals. As long as it does this within all international legal frameworks, there is no ground for penalizing it. Additionally, China is taking other measures that benefit its export sector, such as the tax rebates, energy subsidies and - ironically for a `communist' country - the suppression of effective worker organization in independent labor unions. On these issues US foreign policy has been much less vocal.

The hyperbole in the USA might partly stem from the fact that the USA have very little diplomatic leverage to influence Chinese monetary policy. As our discussion has shown, the only promising way forward would be to seek new multilateral agreements banning monetary policy that penalizies currency manipulation. Such efforts have not been followed seriously by US foreign policy however.

 