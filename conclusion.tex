\section{Conclusion}
\label{sec:conclusion}

When we Look at the economic case against China, the evidence that the 
RMB has indeed been kept undervalued is overwhelming. Chinese monetary 
policy has been adjusting every lever suited to keep their currency at a 
low value. In addition, even though economic measures trying to estimate 
the deviation from the equilibrium exchange rate vary, they almost all 
find that the Chinese RMB in fact is undervalued.

We think that many attempts to point out other factors that also had 
helped to boost the Chinese export miracle over the last decade are 
justified in their assessment. These factors might help explain the 
success of Chinese exports in recent years. However, they do not rule 
out that \emph{in addition} China has been keeping the RMB undervalued.

Providing a definitive proof that China has been engaged in such a 
behaviour is near impossible. For every part of China's monetary policy 
one can find good reasons, so it is difficult to point a finger at a 
specific measure. However, when looking at the exchange rate policy 
China has been conducting over the last decade as a whole we don't know 
how one could have a more clear case of an attempt to undervalue a 
currency.

And yet we find the political debate in the USA to be hyperbolic. When 
US politicians eagerly point their fingers at a supposed wrongdoing they 
ignore the fact that China is not doing anything wrong, i.e. they are 
not violating any legal framework. Every country is using monetary 
policy to satisfy its domestic goals so unless China has signed a treaty 
wowing not to undervalue its currency, it cannot be penalized in doing 
so.  

We have discussed how some economists argue that China by joining the 
WTO has indeed signed a treaty that disallows them from pursuing an 
undervaluation of the RMB.  This is also the argument made by Romney 
during the presidential election in 2012. If this were indeed the case 
then the US would be within their right to pursue China legally within 
the WTO. The fact that the USA has not attempted to use WTO to stop 
China from undervaluing its currency is a strong indicator that the 
illegitimacy of China's monetary policy is not that obvious. The 
hyperbole in the USA might partly stem from the fact that the USA have 
very little diplomatic leverage to influence Chinese monetary policy. As 
our discussion has shown, no promising unilateral nor multilateral 
diplomatic option within existing legal frameworks is available. 

The accusation that China is violating not a legal but a \emph{moral} 
framework of economic fairness is a different case. It might be honestly 
put forward by US labor unions, but as a political argument it would 
only be credible if the same US politicians would push for a 
multilateral treaty banning such behaviour. If US policymakers find 
China's policies to be harmful to such a degree that international 
action should be taken then we would argue that the only promising way 
forward would be to seek new multilateral agreements banning monetary 
policy that penalizes currency manipulation. Such efforts have not been 
followed seriously by US foreign policy however, probably because the 
USA themselves fear to loose the ability to decide their own monetary 
policy autonomously. 

Moreover, China is taking other measures that benefit its export 
sector, such as the tax rebates, energy subsidies and --- ironically for 
a `communist' country --- the suppression of effective worker 
organization in independent labor unions. On these issues US foreign 
policy has been much less vocal.

As long as no multilateral efforts for new international legal 
frameworks are taken, the domestic political debate in the US centers 
around an idea of economic fairness that does not have have any 
application in international economics. As such it does not serve as 
anything else than a rhetoric tool.% to garner votes.


%What's more, the effects of Chinese undervaluation on the American 
%economy is hard to substantiate.  Manufacturing jobs have been 
%disappearing for decades partly due to automatization, and arguing that 
%that every job appearing in China is a loss of an American job is a 
%simplistic version of reality that doesn't take into account how the 
%influx of cheap Chinese goods stimulate the American economy.


%war as well as a multilateral agreement would be disadvantageous for the 
%US in the long run. This leaves US politicians with a case where they 
%can't do much besides promising to take action, a case perfectly suited 
%for a presidential election.
