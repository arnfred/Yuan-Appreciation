\section{Conclusion}
\label{sec:conclusion}

Looking at the economic case against China the evidence that the RMB has 
indeed been kept undervalued by the Chinese national bank is 
overwhelming. Chinese monetary policy has been adjusting every lever suited to keep their currency at a low value.  Even though 
economic measures trying to estimate the deviation from the equilibrium exchange rate vary, they almost 
all find that the Chinese RMB in fact is undervalued.

We think that many attempts to point out other factors that also had helped to boost the Chinese 
export miracle over the last decade are justified in their 
assessment. These factors might help explain the success of 
Chinese exports in recent years. However, they do not rule out that \emph{in addition} China has been keeping the RMB undervalued.

Providing a definitive proof that China has been engaged in such a 
behaviour is near impossible. And because for every part of China's monetary policy one can find good reasons, it is difficult to point a finger at a specific measure. However, when looking at the exchange rate policy 
China has been conducting over the last decade as a whole we don't know 
how one could have a more clear case of an attempt to undervalue a currency.

This doesn't mean that politicians in the US deserve praise for their 
efforts in labeling China as a currency manipulator. When the 
participating politicians eagerly point their fingers at a supposed 
wrongdoing they ignore the fact that China isn't doing anything wrong.
Every country is using monetary policy to satisfy its domestic goals, so
unless China has signed a treaty wowing not to undervalue their 
currency, they are within their right to decide undervalue their 
currency. The domestic political debate in the US seem to center around 
an idea of economic fairness which doesn't has any application in 
international economics and as such doesn't serve as anything else than 
a rhetoric tool to garner votes.

Additionally, China is taking other measures that benefit its export 
sector, such as the tax rebates, energy subsidies and --- ironically for 
a `communist' country --- the suppression of effective worker 
organization in independent labor unions. On these issues US foreign 
policy has been much less vocal. 

What's more, the effects of Chinese undervaluation on the American 
economy is hard to substantiate.  Manufacturing jobs have been 
disappearing for decades partly due to automatization, and arguing that 
that every job appearing in China is a loss of an American job is a 
simplistic version of reality that doesn't take into account how the 
influx of cheap Chinese goods stimulate the American economy.

We have discussed how some economists argue that China by joining the 
WTO has indeed signed a treaty that disallows them from pursuing an 
undervaluation of the RMB. This is also the argument made by Romney 
during the presidential election in 2012. If it were indeed the case 
then the US would be within their right to pursue China legally within 
the WTO. The fact that the US hasn't attempted to use WTO to stop China 
from undervaluing their currency is a strong indicator --- we find --- 
that the illegitimacy of the issue might be less obvious.

If China's policies are indeed unfair to such a degree that 
international action should be taken then the we would argue that the 
only promising way forward would be to seek new multilateral agreements 
banning monetary policy that penalizes currency manipulation. Such 
efforts have not been followed seriously by US foreign policy however, 
something which isn't so surprising. If a country were to sign a treaty 
banning many of the practices that China has been undertaking, this 
would mean a big loss of the ability of said country to decide their own 
monetary policy autonomously.

The hyperbole in the USA might partly stem from the fact that the USA 
have very little diplomatic leverage to influence Chinese monetary 
policy. As our discussion has shown, both a tariff and the ensuing trade 
war as well as a multilateral agreement would be disadvantageous for the 
US in the long run. This leaves US politicians with a case where they 
can't do much besides promising to take action, a case perfectly suited 
for a presidential election.
