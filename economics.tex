\documentclass[11pt]{article}



\usepackage{natbib}
\usepackage{hyperref}
\usepackage{eurosym}

\begin{document}

%first draw, fmolo, 28.10.2012
\section{economics}

\subsection{currency exchange}

Money is a tradable good and different currencies can be traded against each other. Market participants such as private persons, corporations, commercial banks and national banks can exchange a certain amount of a currency for another amount of another currency. The price of a currency A in terms of currency B is called the exchange rate of A and B. For example, as of october 28 2012, you get \textdollar{1.29} in exchange for \euro{1}. The exchange rate is determined by supply and demand for a currency. If at some point the demand for more US dollar rises, for example because a international corporation invests in the US and pays workers there a wage in US dollars, the price of the dollar on the currency market will be higher, i.e. you will get fewer US dollars for one euro.

In the currency market, national banks play a special role. In principle, each national bank has an unlimited supply of its own currency, because they can - figuratively speaking - print a discretionary amount of money in their own currency.\footnote{The process is somewhat more complicated than printing bank notes, but the effect is the same for the purposes of this section.} A national bank can therefore influence the exchange rate of its currency against other currencies. If the national bank of the US, the US Federal Reserve decides to print more US Dollars and uses them to buy euros, the price of the dollar in terms of euro depreciates, i.e. you will get more US dollars for one euro. This process is very normal: Some national banks even use their money supply to `peg' their currency to another. For example, the Swiss National Bank offers every vendor of an euro CHF 1.20 in exchange - and it never runs out of CHF. The exchange rate of the Swiss franc and the euro will therefore never be lower than 1.20, while it can raise higher. As another example, the national bank of Denmark goes even beyond this: It controls the supply of Danish kronor so that the exchange rate of the kronor and the euro constantly remains at 0.134 (with a small bandwith of +/-2.25\%). 

Such practices seem to be accepted in public discourse and by the relevant multilateral agency, the International Monetary Fund. Meanwhile, China has been accused by prominent US politicians of `manipulating' its currency and keeping the Chinese currency, the Renminbi\footnote{abbreviated to CNY. The basic unit of the Renminbi is the Yuan.} `undervalued'. The next section will analyze how this alleged manipulation takes place and what it means for a currency to be undervalued. 

\subsection{undervaluation}

A low currency is an advantage for the export sector of a country. Goods produced in China have lower price if the Renminbi is low, since manufacturers use the Chinese curreny to pay wages, buy components and other supplies. If a manufacturer can cheaply buy Yuan with a fixed amount of US dollars, he can produce more cheaply in China than in the US. 

Chinese goods are very competitive on the world market. Therefore, one might suspect, that at least part of that competitiveness is due to the (too?) low value of the Yuan. 

There is no reliable method to determine if a currency has the `right' value. One common technique is called Purchasing Power Parity (PPP): It defines a set of typical goods and services and asks how much money would be needed in two countries with their own currency to buy the same products in each country. The default case would be when the same amount of money should be necessary to buy the same goods in both countries - this case is used to calculate an `ideal' undistorted exchange rate. 
%check krugman, "introduction to macroeconomics"
%then need to find some data RMB-USD PPP-adjusted

Besides relative price levels, there is another indicator for a misvalued currency: trade imbalances. A country's \emph{balance of trade} is the monetary value of its exports minus its imports, i.e. gains from exports minus payments for imports. Chinas trade balanced peaked at a value of 10\% of its GDP (the sum value of all goods and services produced in China) in 2007, meaning that China exported much more goods and services than it imported. 


\subsection{capital controls}

\subsection{the situation in 2010}

\subsection{the situation by now (2012)}


\end{document}