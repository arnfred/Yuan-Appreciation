%TODO: Where to put in China's export surplus?
\section{Economics}
\label{sec:economics}

The exchange rate between two currencies is based on how much it 
costs to buy a fixed amount of one currency with the other. This price 
depends on the demand and the supply of each currency on the money market. 

Macroeconomic theory postulates, that for every two currencies at every 
moment, there exists a balanced exchange rate, called the 
\emph{equilibrium exchange rate}.\footnote{\cite[pp. 
505]{Krugman2008}.} There is a number of different theoretical methods 
for defining the equilibrium exchange rate. We can broadly group them in 
two approaches: 

\begin{itemize}
\item{Estimates based on purchasing power parity (PPP)}
\item{Estimates based on macroeconomic balance ideals}
\end{itemize}

The basic idea of PPP is to declare the exchange rate as ideal, at which 
with a given amount of money you could buy exactly the same goods in two 
countries. A famous example of such an approach is The Economist's Big 
Mac Index.\footnote{http://www.economist.com/markets/Bigmac/Index.cfm} 
If a Big mac costs 1 USD in the US and 5 RMB in China, the ideal 
exchange rate would be 1 to 5. You can then calculate the deviation of 
the actual exchange rate to the ideal equilibrium exchange rate. Of 
course this is also done with a more diverse basket of goods than a Big 
Mac. 

The second group takes a macroeconomic indicator as a basis, and 
measures how much a country diverges from a level that is considered 
normal. The most prominent of such features is current account. The 
current account is the difference between a country's total exports and 
its total imports. If a country exports more than it imports it has a 
current account \emph{surplus} - China, Germany, Japan and Switzerland 
are all current account surplus countries. If a country imports more 
goods and services than it exports, it has a current account 
\emph{deficit} - the USA are currently the most prominent and by far the 
biggest deficit country. To define a normal level some economists pick 
an absolute balance, e.g. zero. Others, based on the difference between 
emerging and advanced economies allows for a fluctuation of around 3\% 
as to lie within normalcy. The equilibrium exchange rate is now defined 
as the exchange rate, at which the current account would move down (or 
up) to the previously-defined normal level. This idea of an ideal 
exchange rate becomes important when politicians talk about China as a 
``currency manipulator''.

If the market for currencies was working like a standard market of 
goods, with firms and private persons being able to buy and produce 
money, all exchange rates would always be at their equilibrium rate. 
However, money is not a normal tradable good, and the production of 
money for every country is controlled by their national bank. A national 
bank can print a discretionary amount of money in their own 
currency.\footnote{The process is somewhat more complicated than 
printing bank notes, but the effect is the same for the purposes of this 
section.} But how much money should a national bank print? This decision 
is a matter of \emph{monetary policy}. 

\subsection{Monetary policy}

The standard monetary policy of western countries is to define a target 
for inflation, normally around 2\%. The national banks are mandated to 
control the supply of money such that this target is met. Currencies of 
these countries are then freely traded and their exchange rate 
fluctuates with varying demand. 

Nations can also chose to exercise a tighter control of the value of 
their exchange rate. This is very common: Some national banks even use 
their money supply to `peg' their currency to another, so that exchange 
rates are fixed. For example, the Swiss National Bank (SNB) offers 
every vendor of an euro CHF 1.20 in exchange. Since the SNB controls 
the money supply of Switzerland, it will never run out of CHF. As a 
consequence the euro will never be lower than 1.20 until the SNB changes 
its exchange rate policy. In this case monetary policy becomes an 
\emph{exchange rate policy}: Instead of focusing on inflation, the goal 
of the policy is to control the exchange rate.

A strong incentive for a national bank to maintain a \emph{low} exchange 
rate is to boost exports. Take a typical tradable good as an example: A 
watch produced in Switzerland sells at a fixed price in Swiss Francs 
(CHF). If for one euro you get 1.50 CHF exchanges for 0.7 Euro, a watch 
worth 1000 CHF costs 666 Euro in Germany. If the value of the Swiss 
Franc rises, so that one CHF, the price of the same watch is now 1000 
euro in Germany --- this watch is now a much less competitive good on 
the world market.

%fmolo: This historical episode does not really fit in the theoretical 
%part, I think. Jonas: I disagree, I think it introduces concepts that 
%are really important later on
However this behaviour forces trading competitors to take similar steps 
in order to protect their own exports, which easily leads to a situation 
where countries are competing to devaluating their currencies in order 
to export more, a policy known as ‘beggar thy neighbour’. After such an 
episode during the Great Depression in the 1930's, the behaviour was 
internationally recognized as nonbeneficial for all partners involved 
and international institutions were instantiated to create a set of 
rules and enforce that these rules are enforced by all countries that 
are members of the respective organizations. The most prominent of these 
today are the IMF (International Monetary Fond), the WTO (World Trade 
Organization) and - in Europe - the EU (European Union).

With the notions of the equilibrium exchange rate and monetary policy in 
mind we can now restate what it means for a currency to be undervalued: 
A currency is undervalued when its exchange rate is held below the 
equilibrium exchange rate by a national bank by means of monetary 
policy.

\subsection{Is China manipulating the RMB exchange rate?}

Chinese goods have been very competitive on the world market during the 
last ten years and Chinese exports have more than quadrupled since China 
joined the WTO in 2001.\footnote{\cite{FRED2012}.} It is this stunning 
growth that has prompted accusations against China: The People's Bank of 
China have followed a policy that helped keeping the Chinese currency, 
the renminbi (RMB) undervalued.\footnote{The renminbi is also 
abbreviated to CNY, reflecting the name of the basic unit of the 
Renminbi, the Yuan.}

\subsubsection{Arguments in favor}

Some critics of China's monetary policy make argument that the RMB is 
undervalued by pointing out that the Chinese national bank did 
everything a bank would do were it manipulating the exchange rate. 

So how would a National Bank keep a rate below its equilibrium rate? According to 
textbook economics this can be done in three ways:\footnote{\cite[pp. 
514]{Krugman2008}}

\begin{enumerate}
\item{The government can shift supply and demand for its currency by 
	intervening on the foreign exchange market. By buying foreign 
exchange and selling the local currency the government can drive the 
price of foreign exchange up and the price of the local currency down.}
\item{The government can shift supply and demand by means of monetary 
	policy, namely by keeping interest rates low. Lower interest rates 
mean lower returns for foreign investors. If foreign investors refrain 
from investing locally, the demand for the local currency decrease. 
which drives the price of the local currency down.}
\item{The government can impose foreign exchange controls, forbidding 
	foreigners to buy the local currency. This reduces the practical 
demand by outlawing it, which makes the currency price go down.}
\end{enumerate}

According to Goldstein and Lardy this is exactly what the People's Bank 
of China has been doing for a decade:\footnote{\cite[pp. 
40]{Goldstein2008}.}

\begin{enumerate}
\item{The Chinese government has intervened on the foreign currency 
		market on a massive scale: It has been buying foreign 
		currencies, mainly US Dollars (in the form of US government 
		debt) in exchange for RMB to the amount of 10\% of its GDP, i.e. 
		10\% of the value of all goods and services produced in China.} 
	\item{Interest rates in China are relatively low: When the interest 
		rate is adjusted for inflation, the so called \emph{real} 
	interest rate has actually been negative for the most part since 
2006.}
\item{China imposes foreign exchange controls that limits the amount of 
	RMB that is sold to international investors or other governments.}
\end{enumerate}

As a result, critics of China's exchange rate regime say, China's export 
sector has become unfairly competitive. 

Adding further to the pile of evidence, critics argue, is China's 
practice of \emph{sterilization}.\footnote{\cite{Humpage2010} and \cite[p. ?]{Wang2011}.} If the Chinese government buys foreign 
currency paying with RMB, it is increasing the amount of RMB in 
circulation in the economy.\footnote{In economical jargon it is 
expanding the \emph{monetary base}, what (other things equal) leads to 
an increase in money supply} According to standard economic models an 
increase in the money supply means that the currency has less value and 
(all other things equal) prices go up, leading to 
inflation.\footnote{\cite[pp. 432]{Krugman2008}.}
As a result, goods produced in China would become more expensive on the 
world market not due to currency appreciation, but because production 
costs (e.g. wages of Chinese workers) rise with inflation. According to 
this model, even though the People's Bank of China (PBC) keeps the 
\emph{nominal} exchange rate fixed, the \emph{real} exchange rate would 
increase.\footnote{\cite[p. 509]{Krugman2008}.} Inflation would offset the 
competitive advantage of Chinese goods in the long run and cancel out 
the effects of the lower exchange rate for the RMB.

The People's Bank of China is taking measures that are suited to prevent 
a appreciation in real terms: It is \emph{sterilizing} the money inflow, 
mainly by raising reserve requirements for Chinese commercial banks. 
Raising reserve requirements limits the amount of money the commercial 
banks can issue as loans, therefore 'extracting' money out of the 
economy. This in turn limits inflation and prevents the \emph{real} 
value of the RMB to rise. 

By raising reserve requirements, China has prevented about 40\% of the money 
inflows of entering the monetary base since 2003.\footnote{\cite{Humpage2010}.}%TODO: display data?

Despite these measures, China has seen some inflation during the last 
ten years.\footnote{\cite{Humpage2010}.} Even so, China clearly has been 
taking measures to fight inflation and was thereby lowering its 
\emph{real} exchange rate. Accordingly, while inflation should 
compensate an undervalued nominal exchange rate in the long run, China 
is accused of procrastinating `the long run' by preventing inflation to 
happen.

To sum up, critics argue that China is enacting policies that keep the 
RMB undervalued on two levels. The nonimal exchange rate is kept low by 
economic policy which usually causes inflation making the real exchange 
rate rise. To prevent this, China is negates this effect by sterilizing 
the money inflow, thus doubly preventing the RMB from gaining value.

\subsubsection{Arguments against}

There are a variety of arguments supporting the legitimacy of Chinese 
monetary policy. One stance is to deny the importance of the exchange 
rate. A different approach is to deny that China is indeed manipulating 
the exchange rate in the first place.

The first mentioned argumentation points out that China's export success 
can be explained with other factors than the exchange rate. Different 
authors point to different reasons: 

One factor is \emph{labor arbitrage}:\footnote{This factor was pointed at 
by Xu Mingqi of the Institute of World economy of the Shanghai Academy 
of Social Sciences in a talk to our class on September 4 2012.} Chinese 
workers are willing to work at lower wages than workers in importing 
countries. Importantly, accepted wages are not only lower in absolute 
terms but also in terms of purchasing power: A typical wage in China 
allows for a lower standard of living than a typical wage in an 
industrial country, thereby allowing Chinese firms to produce with much 
lower labor costs, both absolute and relative. 

Another explanation is that Chinese producers profit from very cheap 
energy and land rents.\footnote{\cite[pp. 25]{Huang2010}.} These markets 
are not liberalized and prices can therefore be strongly influenced by 
government policy. For Chinese officials on the local as well as on the 
federal level GDP growth is a major ambition. This makes them maintain 
energy and land use prices that are cheaper on average than in 
industrial countries as well as other emerging economies.

Another factor explaining strong Chinese exports has been introduced 
in 2005 by Ben Bernanke, shortly before he was named chairman of the US 
Federal Reserve:\footnote{\cite{Bernanke2005}.} The \emph{saving glut 
hypothesis}. According to Bernanke a special series of circumstances has 
lead to an exceptionally high saving rate, i.e. the percentage of income 
that is saved. Partly due to the lack of social security institutions 
and to the One-Child-Policy, the saving rate of Chinese households is 
among the highest in the world - in 2007 it was 53\% as opposed to for 
example Switzerland's 17,5\%.\footnote{\cite[pp. 20]{Yang2011} and 
\cite{BFS2012}}. These savings drive down interest rates in China and 
allow the local producers to access very cheap loans, which in turn 
allows them to expand production.\footnote{This explanation is also 
favoured by \cite[pp. 41]{Wyplosz2010} and \cite[p. 65]{Reisen2010}.}

A similar approach is to point out the symmetry of the world market. 
While China is running a current account surplus, the US in turn run a 
current account \emph{deficit}, importing much more goods than 
exporting. According to Wyplosz, this deficit is caused by the low 
saving rate of US households and the budged deficit of the US 
government.\footnote{\cite[pp. 39-40]{Wyplosz2010}.} Instead of trying 
to explain China's export success with the alleged undervaluation of the 
RMB, one could with equal justification explain the US deficit with the 
\emph{over}valuation of the US Dollar. 

What all approaches mentioned have in common is the insistence that 
there is no need to invoke the alleged RMB undervaluation to explain 
China's success on the export market. A different argument does not deny 
the importance of the exchange rate but states that each of the 
practices taken by the PBOC may also have legitimate purposes. China is 
not the only country that pegs its currency to another currency - we 
already mentioned that Switzerland is doing so. A fixed exchange rate 
has the advantage of giving firms and consumers certainty about the 
future value of a currency.\footnote{\cite[p. 515]{Krugman2008}} 
Lowering interest rates to spur growth is also a tool widely used by 
National Banks worldwide. On top of that, limiting inflation is 
considered to be one of the main goals of \emph{any} national 
bank.\footnote{China's capital controls might be the only exception, but 
	their importance pales against that of China's foreign exchange 
reserves.} Finally, many other countries - among them Japan, Germany and 
Switzerland - run current account surpluses that are as big as China's, 
relative to their GDP.

This puts us in a situation where one camp argues that China has strong 
exports driven by an undervalued currency and that the proof of this 
undervaluation lies in the policies implemented by the government. On 
the other side we have a camp that counters these arguments, pointing 
out that there could be plenty of reasons why Chinese exports are 
strong, and that every policy measure implemented by the Chinese 
government is legitimate in it's own right.

If only there were a way to objectively measure by how much a currency 
is under- or overvalued we might be able to stop discussing intentions 
and debating cause and effect. Fortunately many economists have worked 
hard at coming up with just this.

%fmolo: The difference could be this: is monetary policy used for normal 
%goals (like controlling inflation) or is it used to manipulate the 
%currency? for some countries these goals align (switzerland), but not 
%so for china (inflation is high, and they still sell RMB)

\subsubsection{A possibly objective measure}

In order to convincingly argue for or against the China's economic 
policies we need an economic model that objectively tells us \emph{by 
how much} the RMB is under- or overvalued. There is a substantive body 
of research trying to do just that.

The most prominent estimates are based on the macroeconomic balance 
approach described above. In particular a 2008 report by the Internation 
Monetary Fund outlines three macroeconomic balance indicators that can 
be used to estimate the disparity between the real and equilibrated 
exchange rate\footnote{\cite{pp. 1}{Lee2008}.}:

\begin{itemize}
	\item{\textbf{The current account approach} looks at projections of 
		a country's current account and tries to estimate how much the 
	exchange rate would need to be adjusted for it to stabilize within a 
certain level.}
\item{\textbf{The reduced-form equilibrium real exchange rate approach}
	tries to estimate the equilibrium directly based on the annual 
change in the current account as well as a number of trade indicators.}
\item{\textbf{The external sustainability approach} tries to find the 
	exchange rate that would stabilize the change in the current account 
of a country to within a certain level.}
\end{itemize}

These techniques have been used by Cline and Williamson in 
their yearly policy brief on equilibrium exchange rates of the Peterson 
Institute for International Economics, a private research institution 
whose researchers have taken a leading role in criticizing Chinese 
exchange rate policy during the last decade.\footnote{\cite{Cline2009} 
and \cite{Cline2012}.} Their estimates are based largely on the first 
and third methods proposed by the IMF. They designate that debt and 
trade surplus above 3\% of GDP is abnormal and try to calculate how much 
the exchange rate would have to change to bring the current account 
within a normal treshold. In 2009 their results showed that the Chinese 
RMB was undervaluated by 21.4\%. Additionally they found the US dollar 
to be 17.4\% percent overvaluated, futher contrasting the value gap 
between the two currencies.

However, in a similar approach using international trade flows Cheung, 
Chinn and Fuji find diverging results for 
2009.\footnote{\cite{Cheung2009}.} According to their calculations the 
RMB might be undervalued by as much as 50\%, but they do not find the 
deviation to be statistically significant and being highly volatile for 
small variations in the data. In a newer paper in 2010, reviewing 
several different estimates they stress the fact that large 
uncertainties surround all the estimates.\footnote{\cite{Cheung2010}.}

Estimates based on the PPP approach again deliver varying results, from 
the RMB actually being \emph{over}valued by 5\% to a 20\% undervaluation 
for 2009, according to Cheung, Chinn and Fuji.\footnote{\cite[pp. 
82]{Cheung2010}.} Other researchers also find huge swings in different 
estimates, with the RMB being undervalued by up to 
47\%.\footnote{\cite[pp. 72]{Subramanian2010}.} One problem of PPP 
approaches is, that the price level in emerging economies is generally 
lower than in advanced economies. According to the so called 
Balassa-Samuelson effect, this is mainly because services and 
non-tradable goods (e.g. a haircut or a restaurant meal) are much 
cheaper in emerging economies.\footnote{\cite[pp. 57]{Frankel2010}.} 
There are attempts to take account of such effects and calculate a 
revised PPP equilibrium exchange rate, but the results are mixed: The 
above-mentioned variations depend largely on different assumptions in 
the calculating models.

The range of estimates --- from 5\% overvalued to 50\% undervalued --- 
clearly illustrates why estimating an equilibrium exchange rate is a 
difficult problem. How we pinpoint the equilibrium and according to what 
factors we measure the divergence is a question which is vague enough 
that these measures lose their objective quality and become just another 
tool in the debate.

%TODO: Mention the complaints of other East Asian countries somewhere. 
%e.g. Herrero/Koivu in Evenett
