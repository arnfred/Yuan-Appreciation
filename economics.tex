\documentclass[11pt]{article}



\usepackage{natbib}
\usepackage{hyperref}
\usepackage{eurosym}

\begin{document}

%first draw, fmolo, 28.10.2012
\section{economics}

%\subsection{world market} basic idea: prices of goods. exports and imports. trade balances. current account. financial account. they depend on the following factors: x, y, z. one of these factors is the exchange rate. or: surplus would tend to 0 

\subsection{foreign exchange market}

Money is a tradable good and different currencies can be traded against each other in the foreign exchange market. Market participants such as private persons, corporations, commercial banks and national banks can exchange a certain amount of a currency for another amount of another currency. The price of a currency A in terms of currency B is called the exchange rate of A and B. For example, as of october 28 2012, you get \textdollar{1.29} in exchange for \euro{1}. The exchange rate is determined by supply and demand for a currency. If at some point the demand for more US dollar rises, for example because a international corporation invests in the US and pays workers there a wage in US dollars, the price of the dollar on the currency market will be higher, i.e. you will get fewer US dollars for one euro.

In the currency market, national banks play a special role. In principle, each national bank has an unlimited supply of its own currency, because they can - figuratively speaking - print a discretionary amount of money in their own currency.\footnote{The process is somewhat more complicated than printing bank notes, but the effect is the same for the purposes of this section.} A national bank can therefore influence the exchange rate of its currency against other currencies. If the national bank of the US, the US Federal Reserve decides to print more US Dollars and uses them to buy euros, the price of the dollar in terms of euro depreciates, i.e. you will get more US dollars for one euro. This process is very common: Some national banks even use their money supply to `peg' their currency to another, so that exchange rates are fixed. For example, the Swiss National Bank (SNB) offers every vendor of an euro CHF 1.20 in exchange. Since the SNB controls the money supply of Switzerland, it will never run out of CHF and the exchange rate of the Swiss franc and the euro will therefore never be lower than 1.20 until the SNB changes its exchange rate policy. As another example, the national bank of Denmark controls the supply of Danish kronor so that the exchange rate of the kronor and the euro constantly remains at 0.134 (with a small bandwith of +/-2.25\%). 

Such practices seem to be accepted in public discourse and by the relevant multilateral agency, the International Monetary Fund. Meanwhile, China has been accused by prominent US politicians of `manipulating' its currency and keeping the Chinese currency, the Renminbi\footnote{abbreviated to CNY. The basic unit of the Renminbi is the Yuan.} `undervalued'. The next section will analyze how this alleged manipulation takes place and what it means for a currency to be undervalued. 

\subsection{nominal undervaluation, currency manipulation}

Macroeconomic theory postulates, that for every two currencies at every moment, there is an equilibrium exchange rate. The equilibrium exchange rate ist determined by supply and demand for each currency in the foreign exchange market. The accusation against China of `manipulating' its currency can therefore be restated: It claims that China is keeping a fixed exchange rate \emph{below} the equilibrium rate. According to textbook economics this can be done in three ways:\footnote{\cite[pp. 514]{Krugman2008}}

\begin{enumerate}
\item{The government can shift supply and demand for its currency by intervening on the foreign exchange market. Buying foreign exchange and selling the local currency drives the price of foreign exchange up and that of the local currency down.}
\item{The government can shift supply and demand by means of monetary policy, namely by keeping interest rates low. Lower interest rates mean lower returns for foreign investors. If foreign investors refrain from investing locally, the demand for the local currency decreases, driving the price of the local currency down.}
\item{The government can impose foreign exchange controls, forbidding foreigners to buy the local currency, therefore again reducing demand and therefore the price of that currency.}
\end{enumerate}

Why would a government do such things? Goods produced in a country with a low exchange rate are cheap relative to goods produced in other countries, since production costs are paid in the (low-valued) local currency. A low exchange rate therefore increases competitiveness of the export sector. 

%normally, the following would happen (Krugman pp. 507-509, and Humpage: http://www.clevelandfed.org/research/trends/2010/1110/01intmar.cfm)
% nominal undervaluation by selling RMB -> increase of RMB money supply -> inflation -> real (instead of nominal) RMB appreciation.
% but it does not happen because china "sterilizes" the money, does not let money supply increase

\subsection{China criticism}

But how do we know a currency is indeed undervalued? The postulated equilibrium interest rate is a virtual value not realized in the foreign exchange market and can not be measured. Indeed there is no reliable method to determine the `right' exchange rate of a currency.\footnote{among others: \cite{pp. 4}{GoldsteinLardy2008}}
%Purchasing power parity (PPP) would possibly be a method, but i have not seen it discussed in relation with China yet.

Critics of China therefore base their case on circumstantial evidence rather than on hard empirical methods. %\footnote{although such studies exist too, \cite{}???} 
According to their claims, China has been doing exactly what textbook economics tell us a government keeping its currency undervalued would do:\footnote{\cite[pp. 40]{GoldsteinLardy2008}}

\begin{itemize}
\item{The Chinese government has intevened on the foreign currency market on a massive scale: It has been buying foreign currencies, mainly US Dollars (in the form of US government debt) in exchange for RMB to the amount of 10\% of its GDP, i.e. 10\% of the value of all goods and services produced in China.} %show data. but first find data (IMF? FRED?)
\item{Interest rates in China are relatively low, with real (i.e. adjusted for inflation) interest rates actually being negative for the most part since 2006.} %show data. but first find data (World Bank?)
\item{China imposes foreign exchange controls that prevent international investors or other governments to buy RMB.}%this seems clear, but i have not yet found how they do it exactly
\end{itemize}

\emph{As a result}, critics of Chinas exchange rate regime would say, China's export sector has become extremely competitive. Indeed, China's exports exceed its imports by far; in absolute terms, such a current account surplus (i.e. the amount by which the value of exports exceed the value of imports) is unprecedented, though not so much in relative terms. %account surplus data is easily available. including switzerland might be fun, because the Swiss surplus is even higher. 

%is monetary policy used for normal goals (like controlling inflation) or is it used to manipulate the currency? for some countries these goals align (switzerland), but not so for china (inflation is high, and they still sell RMB)
 

\subsection{China apology}


\subsection{newest developments; the situation by now (2012)}


\end{document}