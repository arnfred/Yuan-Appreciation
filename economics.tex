%TODO: Where to put in China's export surplus?
\section{The economists' viewpoint}

The exchange rate between two currencies is based on how much it 
costs to buy a fixed amount of one currency with the other. This price 
depends on a vast array of factors and can be influenced both by the two 
governments as well as by market forces.  

Macroeconomic theory postulates, that for every two currencies at every 
moment, there exists a balanced exchange rate, called the 
\emph{equilibrium exchange rate}.\footnote{\cite[p.  ?]{Krugman}}. There is a number of various different theoretical methods for defining the equilibrium exchange rate. We can broadly group them in two approaches: 

\begin{itemize}
\item{Estimates based on purchasing power parity (PPP)}
\item{Estimates based on macroeconomic balance ideals}
\end{itemize}

The basic idea of PPP approaches is to declare the exchange rate as ideal, at which you could buy exactly the same goods in two countries. A famous example of such an approach is The Economist's Big Mac Index.\footnote{http://www.economist.com/markets/Bigmac/Index.cfm} If a Big mac costs 1 USD in the US and 5 RMB in China, the ideal exchange rate would be 1 to 5. You can then calculate the deviation of the actual exchange rate to the ideal equilibrium exchange rate. Of course this is also done with a more sophisticated basket of goods than a Big Mac. 

The second group takes some macroeconomic indicator as a basis, most prominently the current account. The current acount is the difference between a country's total exports and its total imports. If a country exports more than it imports it has a current account \emph{surplus} - China, Germany, Japan and Switzerland are all current account surplus countries. If a country imports more goods and services than it exports, it has a current account \emph{deficit} - the USA are currently the most prominent and by far the biggest surplus country. Economists now take the current account and define a `normal' value. Some theories take a current account of zero as normal. Others, based on the difference between emerging and advanced economies take a current account of around 3\% as normal. The equilibrium exchange rate is now defined as the exchange rate, at which the current account would move down (or up) to the previously-defined normal level.

If the market for currencies was working like a standard market of goods, with firms and private persons being able to buy and produce money as a good, 
all exchange rates would always be at their equilibrium rate. However, money is not a normal good, and the 
the production of money for every country is controlled by their 
national bank. A national bank can just print a 
discretionary amount of money in their own currency.\footnote{The 
process is somewhat more complicated than printing bank notes, but the 
effect is the same for the purposes of this section.}. But how much money should a national bank print? This decision is a matter of \emph{monetary policy}. 

\subsection{Monetary policy}

The standard monetary policy of western countries is to define a target 
for inflation, normally around 2\%. The National Banks are mandated to 
control the supply of money such that this target is met. Currencies of 
these countries are then freely traded and their exchange rate 
fluctuates with varying demand. 

Nations can also chose to exercise a tighter control of the value of 
their exchange rate.  This is very common: Some national banks even use 
their money supply to `peg' their currency to another, so that exchange 
rates are fixed.  For example, the Swiss National Bank (SNB) offers 
every vendor of an euro CHF 1.20 in exchange.  Since the SNB controls 
the money supply of Switzerland, it will never run out of CHF and the 
exchange rate of the Swiss franc. As a consequence the euro will never 
be lower than 1.20 until the SNB changes its exchange rate policy. In 
this case monetary policy becomes an \emph{exchange rate policy}: Instead of 
focusing on inflation, the goal of the policy is to control the exchange 
rate.

There are incentives for a national bank to maintain a \emph{low} exchange rate, i.e. to keep its currency at a low value. Take a typical tradable good as an example: A watch produced in Switzerland sells at a fixed price in Swiss Francs (CHF). If for one euro you get 1.50 CHF exchanges for 0.7 Euro, a watch worth 1000 CHF costs 666 Euro in Germany. If the value of the Swiss Franc rises, so that one CHF exchanges for exactly one euro, the price of the same watch is now 1000 euro in Germany - this watch is now a much less competitive good on the world market.

%fmolo: This historical episode does not really fit in the theoretical part, I think.
%However this behaviour forces trading competitors to take similar steps in order to protect their own exports, which easily leads to a situation where countries are compet- ing to devaluating their currencies in order to export more, a policy known as ‘beggar thy neigbour’. After such an episode during the Great Depression in the 1930's, the behaviour was internationally recognized as nonbeneficial for all partners involved and international institutions were instantiated to create a set of rules and enforce that these rules are enforced by all countries that are members of the respective organizations. The most prominent of these today are the IMF (International Monetary Fond), the WTO (World Trade Organization) and - in Europe - the EU (European Union).

With the notions of the equilibrium exchange rate and monetary policy in mind we can now restate what it means for a currency to be undervalued: A currency is undervalued when its exchange rate is held below the equilibrium exchange rate by a national bank by means of monetary policy.

\subsection{Is China manipulating the RMB exchange rate?}


Since manipulating the exchange rate can be beneficial for a nation’s exports and foreign investments, a national bank might be tempted to promote its country’s exports by holding the exchange rate low. The People's Bank of China has been accused of following such a policy and keeping the Chinese currency, the renminbi (RMB) undervalued.\footnote{The renminbi is also abbreviated to CNY, reflecting the name of  the basic unit of the Renminbi, the Yuan.}

\subsubsection{Yes}

Such criticism is based on an argument of the following structure: First
you ask what a national bank would do if 
it were trying to manipulate its currency. Second you point out that the Chinese national bank is doing exactly that. 

So how would a National Bank keep a rate below its equilibrium rate? According to 
textbook economics this can be done in three ways:\footnote{\cite[pp. 
514]{Krugman2008}}

\begin{enumerate}
\item{The government can shift supply and demand for its currency by 
	intervening on the foreign exchange market. By buying foreign 
exchange and selling the local currency the government can drive the 
price of foreign exchange up and the price of the local currency down.}
\item{The government can shift supply and demand by means of monetary 
	policy, namely by keeping interest rates low. Lower interest rates 
mean lower returns for foreign investors. If foreign investors refrain 
from investing locally, the demand for the local currency decreases.  
which drives the price of the local currency down.}
\item{The government can impose foreign exchange controls, forbidding 
	foreigners to buy the local currency. This reduces the practical 
demand by outlawing it, which makes the currency price go down.}
\end{enumerate}

According to Goldstein and Lardy\footnote{\cite[pp.  
40]{GoldsteinLardy2008}}  this is exactly what the People's Bank of 
China has been doing for a decade:

\begin{enumerate}
\item{The Chinese government has intervened on the foreign currency 
		market on a massive scale: It has been buying foreign 
		currencies, mainly US Dollars (in the form of US government 
		debt) in exchange for RMB to the amount of 10\% of its GDP, i.e. 
		10\% of the value of all goods and services produced in China.} 
		%TODO: show data. but first find data (IMF? FRED?)
\item{Interest rates in China are relatively low: When the interest rate 
	is adjusted for inflation, the so called \emph{real} interest rate 
have actually been negative for the most part since 2006.} %show data.  
but first find data (World Bank?)
\item{China imposes foreign exchange controls that limits the amount of RMB that is sold international investors or other governments.}
\end{enumerate}

As a result, critics of Chinas exchange rate regime say, China's export 
sector has become unfairly competitive. 

Adding further to the pile of evidence, critics argue, is China's 
practice of \emph{sterilization}.\footnote{\cite{Humpage2010} and \cite[p. ?]{Wang2011}} If the Chinese government buys foreign 
currency paying with RMB, it is increasing the amount of RMB in 
circulation in the economy.\footnote{In economical jargon it is 
expanding the \emph{monetary base}, what (other things equal) leads to 
an increase in money supply}. According to standard economic 
models\footnote{\cite[pp. ?]{Krugman2008}} an increase in the money 
supply means that the currency has less value and prices go up, leading 
to inflation.%TODO: Maybe quickly explain the assumed mechanism?} 
As a result, goods produced in China would become more expensive on the 
world market not due to currency appreciation, but because production 
costs (e.g. wages of Chinese workers) rise with inflation. According to 
this model, even though the People's Bank of China (PBC) keeps the 
\emph{nominal} exchange rate fixed, the \emph{real} exchange rate would 
increase.\footnote{\cite[p. 509]{Krugman}} Inflation would offset the 
competitive advantage of Chinese goods in the long run and cancel out 
the effects of the lower exchange rate for the RMB.

%TODO: Just a note: Humpage2010: http://www.clevelandfed.org/research/trends/2010/1110/01intmar.cfm

Again, the People's Bank of China is taking measures that are suited to prevent a appreciation in real terms: It is \emph{sterilizing} the 
money inflow, mainly by raising reserve requirements for Chinese commercial banks. Raising reserve requirements limits the amount of money the commercial 
banks can issue as loans, therefore 'extracting' money out of the economy. This 
in turn limits inflation and prevents the \emph{real} value of the RMB to rise. 

By raising reserve requirements, China has prevented about 40\% of the money 
inflows of entering the monetary base since 2003.\footnote{IMF, via Cleveland Fed, 
http://www.clevelandfed.org/research/trends/2010/1110/01intmar.cfm}%TODO: display data?

Despite these measures, China has seen some inflation during the last ten years. But inflation in China 
has only been moderatly higher than in other countries. In fact the real 
and the nominal exchange rate against the US Dollar roughly moved in unison until the year 2010.\footnote{source: 
http://www.clevelandfed.org/research/trends/2010/1110/01intmar.cfm}%TODO: find data to display

Accordingly, while inflation should raise the nominal exchange rate in the long run, China is accused of procrastinating `the long run' by preventing inflation to happen.

\subsubsection{No}

There is a variety of arguments in defense of Chinese monetary policy. One group denies the importance of the exchange rate; a second denies that China is indeed manipulating the exchange rate.

First, as pointed out in the first section of this chapter, China's export success can be explained with a variety of factors.  Different authors point to different reasons: Huang Yiping focuses on factor cost distortions.\footnote{Huang2010, http://www.voxeu.org/article/china-us-and-renminbi-rejoinder-krugman} Charles Wyplosz's and Helmut Reisen's focus is on the high saving rate in China.\footnote{\cite[pp. 40]{Wyplosz2010} and \cite{p. 65}{Reisen2010}}. What they have in common is the insistence that there is no need to invoke the alleged RMB undervaluation to explain China's success on the export market.

Similarly, one might point out the symmetry of the world market.\footnote{\cite[pp. 39-40]{Wyplosz2010}.} While China is running a current account surplus, the US in turn run a current account deficit, importing much more goods than exporting. According to Wyplosz, this deficit is caused by the low saving rate of US households and the budged deficit of the US government. Instead of trying to explain China's export success with the alleged undervaluation of the RMB, one could with equal justification explain the US deficit with the \emph{over}valuation of the US Dollar. 

Another line of defense states that each of the practices taken by the POBC may also have legitimate purposes. China is no the only country that pegs its currency to another currency - we already mentioned Switzerland and Denmark doing so. A fixed exchange rate has the advantage of giving firms and consumers certainty about the future value of a currency.\footnote{\cite[p. 515]{Krugman2008}} Lowering interest rates to spur growth is also a tool widely used by National Banks worldwide. And limiting inflation is considered to be one of the main goals of \emph{any} National Bank.\footnote{China's capital controls might be the only exception, but their importance pales against that of China's foreign exchange reserves.}. Finally, many other countries - among them Japan, Germany and Switzerland - run current account surpluses that are as big as China's, relative to their GDP.

%fmolo: The difference could be this: is monetary policy used for normal goals (like controlling inflation) or is it used to manipulate the currency? for some countries these goals align (switzerland), but not so for china (inflation is high, and they still sell RMB)

\subsubsection{Maybe, but how much?}

The equilibrium exchange rate is an elusive concept. The above-mentioned calculations have been heavily critized.

The approaches based on purchasing power parity (PPP) are considered problematic, because they yield such different results.\footnote{\cite[p. 16]{Yu2010} and \cite[pp. 82]{CheungChinn2010}} The result depends heavily the method chosen and its assumptions, too much so as to serve as a guide for exchange rate policy.

Approaches based on macroeconomic balance ideals also face problems: 
%TODO: Name these problems


%fmolo: I integrated this in the following paragraphs. (Later moved further up)
%Similarly we can focus on the 
%purchasing power parity (PPP)\footnote{The PPP is a measure for how the 
%price for similar goods and services in two countries differ}. Based on 
%the behaviour of poor countries in growth based on the PPP, it is 
%estimated that the RMB is undervalued between 12\% and 
%47\%\footnote{\cite{Subramanian2010}}.

% It's really funny that this text (Subramanian2010) is quoting Cline 
% and Williamson as well as Goldstein and Lardy for agreeing on the same 
% number, when in fact Goldstein and Lardy have their number from Cline 
% and Williamson (2009)

% Also from the same paper:
% Since each of the four estimates suffers from limitations, a 
% reasonable approach would be to average all four. This yields an 
% undervaluation estimate for China of about 31\% against the dollar, 
% which is my preferred PPP-based estimate.
% who are giving grants to these people?
%fmolo: awasn't me



A problem of PPP approaches is, that the price level in emerging economies is generally lower than in advanced economies. According to the so called Balassa-Samuelson effect, this is mainly because services and non-tradeable godds (e.g. a haircut or a restaurant meal) are much cheaper in emerging economics.\footnote{\cite[pp. 82]{CheungChinn2010}.} There are attempts to take account of such effects and calculate a revised PPP equilibrium exchange rate, with mixed results: Depending on the detailed assumptions, such attempts result in a wide range from the RMB being \emph{over}valued to a \emph{under}valuation of 47\%.\footnote{\cite[pp. 72]{Subramanian}.} Some attempts reach no statistically significant results.\footnote{\cite[p. 83]{CheungChinn2010}.} Since most such calculations show a undervaluation of the RMB, some economists conclude the RMB must be undervalued, and they then estimate the degree of undervaluation to an average of the different approaches.\footnote{}%does anyone else than Subramanian do that?


%fmolo: I think this needs not to be so detailed. What do you think? I could shorten it.
However, to understand them it's necessary to introduce the notion net 
foreign asset as well as the concepts of current account and foreign 
account\footnote{For a more in depth explanation \cite{ch.  
18}{KrugmanTextbook} provides a good introduction}.

The net foreign asset is the value of the assets that a country owns 
minus the value of assets from that country which is owned abroad.  
Assets in this sense is usually state bonds but can also be stocks and 
goods.  

The current and financial account are measures for how the net foreign asset changes.  
The current account constitutes the balance of trade and money transfers 
while the financial account constitutes the balance of financial assets, 
that is the money borrowed from abroad or the amount of money lent to 
other countries. 
%fmolo: could this be integrated into the "export success story" told in the first section?

The two accounts are related by the current account plus the financial 
accout being equal to zero. This makes sense intuitively since if a 
nation buys more goods than it can finance with exports it needs to 
finance this by borrowing money abroad instead. In this case, the 
negative trade balance translates to a current account deficit, while 
the influx of money coming from borrowing money translates to a finance 
account surplus.

When it comes to estimating the equilibrium exchange rate these three 
measures are heavily used because they gives us an idea of how stable an 
economy is, judging from how assets and goods are flowing in and out of 
the economy. In particular a report was released in 2008 by the 
Internation Monetary Fund outlining three methods that can be used to 
estimate the disparity between the real and equilibrated exchange 
rate\footnote{The Report: \cite{pp.  1}{Lee08}}:

\begin{enumerate}
	\item{The \emph{macroeconomic balance approach} looks at projections 
		of a country's current account and tries to estimate how much 
	the exchange rate would need to be adjusted for it to stabilize 
within a certain level}
\item{The \emph{reduced-form equilibrium real exchange rate approach} 
	tries to estimate the equilibrium directly as a function of the net 
foreign asset as well as a number of trade indicators}
\item{The \emph{external sustainability approach} tries to find the 
	exchange rate that would stabilize the net foreign asset of a 
country to within a certain level}
\end{enumerate}

In practice these techniques have been used by Cline and Williamson in 
their yearly policy brief on equilibrium exchange 
rates\footnote{\cite{cline2009,cline2012}}.  Their estimates are based 
largely on the first and third methods proposed by the IMF. They 
designate that debt and trade surplus above 3\% of GDP is abnormal and 
tries to calculate how much the exchange rate would have to change to 
bring the current account within a normal treshold. In 2009 their 
results showed that the Chinese RMB was undervaluated by 21.4\%, a 
number which has been much quoted since then. Especially in relation to 
the fact that they found the US dollar 17.4\% percent overvaluated, 
futher contrasting the value gap between the two currencies.

%fmolo: i don't understand the method that follows
%jonas: It's the same as above. The PPP method. the two sections should 
%be merged
% 
Instead of trying to find the equilibrium exchange rate, a different 
approach is to do the exact opposite. If we pick a comparative point in 
time or statistical measures based on other countries, we can measure 
how much the current exchange rate deviates from a factor that remains 
constant.

If we pick the unit price of labour as our constant and 1998 as our 
point of reference it is straightforward to show that the RMB is 25 
percent undervalued when compared to at least the American 
Dollar\footnote{\cite{chimerica2009}}. 








\subsubsection{it is over}

The last point to make against the accusations against China is extremely straight-forward: Even if the RMB was undervalued in the past, it is not so any more. Even some of the most ardent critics of China's exchange rate policy now reach this conclusion.\footnote{\cite{Krugman2012}http://krugman.blogs.nytimes.com/2012/10/22/an-issue-whose-time-has-passed/ and \cite{ClineWilliamson2012}} %krugman has graphs but doesn't indicate the source. TODO: show data

%We could therefore turn to the question if this adjustment is due to diplomatic pressure ot the US.


\subsection{Wrap up: Up to 15\% or more}

%on each side there are data-driven arguments. but lack of generally acknowledged models results in varying estimates. their value is hard to assess, but given the uncertainties I find this case against China not convincing.

%then there are arguments arguing with theoretical models/mechanisms. It assesses each element of Chinese XR-policy in itself. For every element there might be legitimate reasons. Taken together I agree with Frankel: It is hard to imagine a clearer case of XR manipulation, even though it is not VERY clear.

%\end{document}

%TODO: Mention the complaints of other East Asian countries somewhere. e.g. Herrero/Koivu in Evenett
