\section{Discussion}
\label{sec:discussion}


Seeing that one decade of intense discussion hasn't answered the 
question of whether China should appreciate their currency, so it might 
be a little ambitious to aim for a definitive conclusion on this matter.  
Instead, what we set out to do in the following part is two-fold. We 
will first try to look at how the viewpoints from the two sides of the 
academic debate diverges. Secondly we will assess the political debate 
looking at the motivations to argue one way or another.

\subsection{Academic Tug of War}

At the center stage of the Academic debate we see the discussion 
surrounding the equilibrium exchange rate. Every single way we choose to 
measure this rate shows us that the RMB is severely under appreciated 
creating a strong argument in favor of appreciation. Supporting this 
idea we have Goldstein and Lardy enumerating how the Chinese state is 
guilty of doing everything you would expect a currency manipulator to 
do.

However, the response in defense of China's economic policies points out 
that this critique completely misses the point. Their argument is 
shortly summarized in two parts: 1) The monetary policy of China is a 
reasonable response in light of the economic circumstances and 2) 
Historically other nations such as Vietnam and Japan has been pressured 
into appreciating their currency with negative consequences to follow.

Economically speaking this highlights a clash of economic cultures. One 
the one side we see an economics tradition based on complex modelling 
and theoretical frameworks to understand the world. The idea driving 
this work is that we can deduce economic policies based on a deep 
understanding of the key factors the play a role in macro economics, 
i.e. by estimating a theoretical property such as the equilibrium 
exchange rate, we can deduce the policy that should be implemented.  

The other side of the debate is represented in a larger degree by a 
practical and empirical method. Here the theoretical argument is 
discarded because it is too vague and imprecise. Instead the policy is 
based on using a set of practical tools (sterilization, foreign currency 
accumulation, etc.) used to obtain growth and prosperity. From this 
angle it is much more important to look at past experiences from other 
countries and learn practical lessons from their mistakes and successes, 
than it is to adhere to a theoretical model of how the perfect economic 
policy would look like.

% jonas: What are the real consequence to the US and Chinese economy?


% jonas: Maybe we need a few practical examples of the differences in 
% these two approaches

% jonas: In all cases this argument could be fleshed out more

\subsection{Motivations}

While the theoretical approach can easily look far fetched and out of 
touch with reality, the danger of the practical approach is that it 
might miss the big picture and end up shortsightedly driving the world 
economy towards a sub optimal economic environment. After all, a 
well-functioning domestic economic policy might easily work at the 
expense of neighboring nations, like the classical ``beggar thy 
neighbor'' situations introduced earlier. No man is an island, and 
similarly no economy stands alone in a global market. The question that 
lingers is to what extend a nation should be limited in it's internal 
affairs to the benefit of other countries.

Historically China has been a small player on the world market for most 
of the past century despite its population size. Being a small player 
comes with the advantage that your economic policies affect other 
countries in a very minor degree which permits a wide liberty in setting 
them. China's economy can by no means be called small anymore, but the 
attitude put forth in their economic policies still mirrors those of a 
China that would best like to be left to do as they wish with regards to 
their own internal affairs.

The US on the other hand has historically been promoting free trade with 
the rest of the world, using IMF and WTO as instruments to open up the 
global market partly to their own advantage. Economically speaking the 
US has a history of meddling in the affairs of other countries 
convincing for example Japan and Vietnam to appreciate their currencies 
when it was seen as economically beneficial for the US that they do so.

These extremes set the stage for negotiations where any argument made by 
the opponent side can easily be distorted to reflect the more extreme 
part of their position. If the US tries to find common ground on 
currency speculation they can quickly be labeled as imperialistic, while 
any attempt China makes to retain control over their economic 
development could be categorized as protectionist.

\subsubsection{Manufacturing Jobs}

In his article ``Assessing China's Exchange Rate Regime'', Jeffrey 
Frankel and Shang-Jin Wei\footnote{\cite{Frankel07}} starts by looking 
at the yearly US Treasury Reports. They found that the whether China 
would be labeled as a currency manipulator correlated not just with the 
current account ration as expected but also with metrics that aren't 
usually deemed to have an effect on China's currency policy such as the 
trade balance or the unemployment rate on election years. 

There is no doubt that China to a certain extend serves as a scapegoat 
when manufacturing jobs are disappearing, and not without right.  
Manufacturing jobs are to a large extend being moved away from 
industrialized countries, often because of circumstances similar to 
those that the company Fluttr faced in our fictional example. It is 
however nonsense to expect that raised exchange rates is enough to stop 
this trend. If jobs aren't going to China, there are countries like 
Malaysia, Indonesia and India ready to step in, and if the RMB 
strengthens, those countries are where the jobs will go, and not the US.

With this in mind it's clear that presidential candidates like Obama in 
2008 and Romney in 2012 are merely paying lib service to the large share 
of dissatisfied voter groups that they hope will help them win the 
election. In US domestic policy China is an easy target because invoking 
the Chinese problem evades looking at domestic causes for the same 
problems.

This doesn't mean however that China couldn't potentially gain from 
strengthening the RMB. When Bernanke argues that raising the value of 
the RMB could be a win/win, he is not necessarily merely protecting 
national interests by perpetrating the myth of the disgruntled 
manufacturing industry. As we mention earlier, China needs to take 
strong measures to prevent inflation on their domestic market. On top of 
that the cheap RMB makes it difficult for Chinese people to buy products 
abroad, limiting their experienced prosperity in spite of the staggering 
growth rates.

\subsubsection{Responsibilities}

Academics are free to poke at China with a stick arguing that the RMB is 
under valued according to this or that model, and American politicians 
and manufacturers can complain about the joblosses that China's policy 
is leading to. However they won't get results unless either Chinese 
policy makers decide that it is in their best interest to appreciate the 
RMB or an international organization is able to pressure China into 
doing it anyway. Outside of a modest appreciation between 2006 and 2010 
neither happened despite consistent pressure from the US. Depending on 
your viewpoint this end result can be seen either as a confirmation that 
China never had an obligation to react, or that China neglected to step 
up to their responsibilities as a major economic power.

The idea of a savings glut introduced by the chair of the US Fed Ben 
Bernanke gives a strong argument for the latter by demonstrating how 
Chinese policies create a situation where enormous amounts of liquidity 
is created by the Chinese savings rate which gives an imbalanced 
investment abroad which to a certain extend explains both the current 
account deficits of the states and part of the current account surplus 
experienced by China. Because this savings rate is tied up to domestic 
economic policies in China but has global consequences, it would be 
frivolous to let China get away with manipulating it in ways that hurt 
other economies.

China is naturally not interested in giving up this kind of autonomy, 
and so far they have not seen any initiative that could challenge them 
to that right. In the current situation it is hard to judge if they are 
taking unfair advantage of a broken system, or if they have every right 
to remain sceptic of foreign attempts at influencing their policies.  
However what the conflict shows very clearly is that had it been the 
case that China had been wrongfully taking advantage of their position 
as a major economic player, there would have been no system currently in 
place that would have been able to stop that from happening, outside of 
a full blown tariff war with the US.

In a future scenario it is likely that China will stand at the other 
side of the argument losing manufacturing jobs to poorer nations like 
Indonesia or India and arguing that these countries in turn are 
undervaluating their currency. When that time comes, it will hopefully 
be a little clearer when exactly this kind of policies is actually 
hurtful to the world economy, and not just an opportunity for poorer 
countries to get a foot in the door of mass manufacturing.

