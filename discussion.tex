\section{Discussion}

Seeing that one decade of intense discussion hasn't answered the 
question of whether China should appreciate their currency, so it might 
be a little ambitious to aim for a definitive conclusion on this matter.  
Instead, what we set out to do in the following part is two-fold. We 
will first try to look at how the viewpoints from the two sides of the 
academic debate diverges. Secondly we will assess the political debate 
looking at the motivations to argue one way or another.

\subsection{Academic Tug of War}

At the center stage of the Academic debate we see the discussion 
surrounding the equilibrium exchange rate. Every single way we choose to 
measure this rate shows us that the RMB is severely under appreciated 
creating a strong argument in favor of appreciation. Supporting this 
idea we have Goldstein and Lardy enumerating how the Chinese state is 
guilty of doing everything you would expect a currency manipulator to 
do.

However, the response in defense of China's economic policies points out 
that this critique completely misses the point. Their argument is 
shortly summarized in two parts: 1) The monetary policy of China is a 
reasonable response in light of the economic circumstances and 2) 
Historically other nations such as Vietnam and Japan has been pressured 
into appreciating their currency with negative consequences to follow.

Economically speaking this highlights a clash of economic cultures. One 
the one side we see an economics tradition based on complex modelling 
and theoretical frameworks to understand the world. The idea driving 
this work is that we can deduce economic policies based on a deep 
understanding of the key factors the play a role in macro economics, 
i.e. by estimating a theoretical property such as the equilibrium 
exchange rate, we can deduce the policy that should be implemented.  

The other side of the debate is represented in a larger degree by a 
practical and empirical method. Here the theoretical argument is 
discarded because it is too vague and imprecise. Instead the policy is 
based on using a set of practical tools (sterilization, foreign currency 
accumulation, etc.) used to obtain growth and prosperity. From this 
angle it is much more important to look at past experiences from other 
countries and learn practical lessons from their mistakes and successes, 
than it is to adhere to a theoretical model of how the perfect economic 
policy would look like.

% jonas: What are the real consequence to the US and Chinese economy?


% jonas: Maybe we need a few practical examples of the differences in 
% these two approaches

% jonas: In all cases this argument could be fleshed out more

\subsection{Motivations}

While the theoretical approach can easily look far fetched and out of 
touch with reality, the danger of the practical approach is that it 
might miss the big picture and end up shortsightedly driving the world 
economy towards a sub optimal economic environment. After all, a 
well-functioning domestic economic policy might easily work at the 
expense of neighboring nations, like the classical ``beggar thy 
neighbor'' situations introduced earlier. No man is an island, and 
similarly no economy stands alone in a global market. The question that 
lingers is to what extend a nation should be limited in it's internal 
affairs to the benefit of other countries.

Historically China has been a small player on the world market for most 
of the past century despite its population size. Being a small player 
comes with the advantage that your economic policies affect other 
countries in a very minor degree which permits a wide liberty in setting 
them. China's economy can by no means be called small anymore, but the 
attitude put forth in their economic policies still mirrors those of a 
China that would best like to be left to do as they wish with regards to 
their own internal affairs.

The US on the other hand has historically been promoting free trade with 
the rest of the world, using IMF and WTO as instruments to open up the 
global market partly to their own advantage. Economically speaking the 
US has a history of meddling in the affairs of other countries 
convincing for example Japan and Vietnam to appreciate their currencies 
when it was seen as economically beneficial for the US that they do so.

These extremes set the stage for negotiations where any argument made by 
the opponent side can easily be distorted to reflect the more extreme 
part of their position. If the US tries to find common ground on 
currency speculation they can quickly be labeled as imperialistic, while 
any attempt China makes to retain control over their economic 
development could be categorized as protectionist.




What is there to discuss. Well; two parts:

Part 1: How are the economic arguments lined up. In particular discuss 
the validity of estimating a real exchange rate, then the purposefulness 
of looking at the exchange rate as the bad factor, and lastly the 
consequences if the exchange rate is actually adjusted, either quickly 
(as advocated by Krugman in 2010) or slowly.

Part 2: Next step is to discuss what the motives behind the US 
argumentation, starting with the domestic debate and integrated the 
remarks from Evenett chapter 7 arguing that the US Treasury report's 
judgement of the chinese RMB statistically correlates with domestic 
factors such as unemployment that shouldn't play a role if the goal was 
to objectively asses the RMB dollar Balance. At the same time we should 
make sure to mention how the argumentation based on the manufactoring 
industry has been repeated, and the idea that the moment China loses a 
job, a job will be created in the US which isn't necessarily true. This 
ties together with a general off-shoring scare.

Then we can talk about china's responsibility in a world where the 
economic choices of one nation influences the choices of another. This 
discussion should touch down on China's past as an insignificant 
econonomy and how they would like to continue pursuing their domestic 
economic policy without anybody getting involved. This leads on to a 
discussion on how the savings glut creates an unbalanced world economy 
and if this fact makes China responsible for acting.
