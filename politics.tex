
\section{The political debate}

The following graph depicts the movement of the RMB real exchange rate against the US Dollar:

%TODO: Insert graph: real exchange rate from ca. 1994-2012

The RMB was pegged to the Dollar from 1994 at a rate of 8.28 until 2005. Then it appreciated at about 6\% per year, before the appreciation came to a halt in the aftermath of the financial crisis. The RMB appreciated again from mid 2010 up to now.

Following this movement of the RMB exchange rate, this chapter will depict the political debates about RMB appreciation in the USA and China.


\subsection{The political debate in the USA}

The history of the political debate in the USA about RMB undervaluation can roughly be seen in four phases:

It started shortly after China had joined the World Trade Organization WTO in 2001. As a member of the WTO, China gained access to Western markets (and vice versa) within an international regulatory structure ensuring free trade and banning protectionist measures. Critics in the USA argued that China was taking unfair advantage of these rules and bending them to China's own advantage. At the core of the critics of Chinese foreign exchange policy were the American labor unions.\footnote{\cite[pp. 14]{Levy2011}} Organized labor in the US has been critical of free trade in general, at least since the North American Free Trade Agreement (NAFTA) had been signed into law in 1993, and due to the ongoing decline of manufacturing in the US. In addition, the decline of products `made in USA'  was contrasted with ever more products labelled `made in China', as a result of China's export surge. Labor unions held the position that this decline was at least partly due to employers shifting production from the US to China. In 2004, the Assistant Director for International Economics at AFL-CIO, the umbrella federation for US unions, stated before the US Congress: 

\begin{quote}
[\dots]it is clear that the Chinese government’s manipulation of its currency, violation of international trade rules, and egregious repression of its citizens’ fundamental democratic and human rights are key contributors to an unfair competitive advantage.\footnote{AFL-CIO on U.S.-China Ties: Reassessing the Economic Relationship, http://www.dossiertibet.it/news/afl-cio-us-china-ties-reassessing-economic-relationship}
\end{quote}

Speaking of currency \emph{manipulation} and an \emph{unfair} competitive advantage, the AFL-CIO Director makes clear that she does not consider China's exchange rate policy as purely technical but as an issue where political action of the USA is needed to protect the interests of American manufacturing workers.

In the early 2000's, labor unions were largely alone in their call for action against China's currency policy. For example, in a 2003 issue of \emph{The international economy magazine} over thirty international analysts from business and academia shortly answered the question `Is China's currency dangerously undervalued and a threat to the global economy?'.\footnote{\cite{IEM2003}.}
Most US analysts held the view, that even if China's currency was undervalued, export surplus did not come at the cost of US manufacturing. In their view, manufacturing was inevitably moving from the US to emerging economies. If the RMB exchange rate had been manipulated, it had been so at the cost of other emerging economies like India or Vietnam. Correspondingly, experts representing other East Asian countries in the \emph{International economy magazine} tended to a more harsh view of China's exchange rate policy. In addition, US experts pointed out the benefits of cheap Chinese exports for American consumers, who could choose from a growing variety of cheap consumer products like toys and electronics. Thus, cheap Chinese exports helped in holding US inflation low.

In 2005 the \emph{saving glut hypothesis} brought forward by Ben Bernanke opend a new line of criticism on China's export surplus.\footnote{as explained in section ...} %TODO: cite section
In this view, the Asian export surpluses posed a threat to global economic stability. However, since this critique came in a time of worldwide economic prospoerity, and since the RMB had started to appreciate against the US Dollar anyway, the critique was not initially drawn into political spotlight.\footnote{says \cite[p.16]{Levy2010}.}

During the \emph{Great Recession}, the financial and economic crisis that started with the burst of the US housing bubble in 2006, the critique of China intensified. As the view prevailed that the crisis was partly due to lack of demand and excess of savings worldwide, prominent economists like Paul Krugman - writing for the New York Times - joined the critics.\footnote{\cite{Krugman2009}.} In addition, the continuous appreciation of the RMB had stopped in 2008. As a result, the critique became more heated on a political level. With each American election, a new batch of politicians paraded their 
toughness on China's economic policies promising tougher measures and 
severe consequences. The most prominent are of course the presidential candidates and  Barack Obama in 
2008\footnote{\cite{Obama2008}.} and Mitt Romney in 2012\footnote{\cite{Romney2012}.} both made a tough stance against China's currency policy part of their agenda, wit both candidates following a remarkably similar argumentation, basing their argument on the trade ramifications of a 
purportedly undervalued RMB. Interestingly, Obama made the above-mentioned statement during a speech to the `Alliance of American 
Manufacturing', an interest organization representing a part of the US 
manufacturing industry. Seemingly, after being joined by monetary policy officials and economists, the labor unions still remain at the core of the critics of China's exchange rate policy.


%This is a popular agenda amongst many Americans who 
%see the consequences of China's entrance on the global manufacturing 
%market in their own lives. For every American product out competed by a 
%cheaper Chinese alternative American jobs are lost, and a candidate who 
%chastises China and vows to take measures to stop this will likely be 
%received well.

The debate in the US reached a climax in 2009 up to the spring of 2010 when the Obama administration, as well as policy experts worldwide were discussing the options for US political action against China's exchange rate policy.


\subsection{The political debate in China}

%Governor of the People's Bank of China, Zhou Xiaochuan, among other PBC 
%officials, is worried about excessive foreign exchange reserves: 
%http://www.wantchinatimes.com/news-subclass-cnt.aspx?cid=1102&MainCatID=&id=20110424000093

Political discussions of the RMB issue in China are of course much less 
transparent than the debate in the USA. It is the State Council in 
Beijing, headed by the Premier Minister - until 2012 this was Wen Jiabao 
- that sets macroeconomic policy, including the exchange rate regime.  
This policy has ultimately to be approved by the Politburo Standing 
Committee, headed by Hu Jintao until 2012. Instead of following the 
debate we will focus on different motivations and interest groups that 
might influence this political sphere.

The interst group most directly concerned by exchange rate policy are 
Chinese exporting firms. Exporters generally have a strong interest in a 
weak RMB, since it makes their products cheaper and thereby more 
competitive on the world market. They are mainly located in eight 
provinces along the Chinese coast. Exporting firms make up only 5-6\% of 
Chinese employment, but their influence is amplified by their enormous 
profits and their close relations to the government, especially through 
politicians of said provinces.\footnote{\cite[p. 202]{Breslin2010}.} 

Also in favor of expansive monetary policy are many representatives of 
local Chinese government. Their main legitimization - and sometimes 
source of income - is economic growth.\footnote{\cite[pp. 
19]{Levy2011}.} Ultimately, this is also the only legitimization of the 
central government in Beijing, giving a strong rationale for 
growth-spurring monetary policy, including a low RMB exchange rate.

As mentioned in the section on economics, one problem in maintaining a 
low exchange rate is inflation. Inflation can be politically disruptive, 
spurring popular discontent. Accordingly, controlling inflation is one 
of the main macroeconomic goals of the central 
government.\footnote{\cite{Naugthon2011}.}

Stability in terms of inflation and growth in forms of exports are 
joined by Chinese nationalism as the main motives behind policy choices.  
The percieved humiliations of the 19th century and Chinas' lackluster 
economic performance up until recent times has fueled a political will 
to reinstante China in it's rightful place\footnote{Much more can be 
read in \cite{hughes2006}}. This has lead to a political atmosphere 
where it's important not to appear weak in confrontations with western 
powers.

In terms of economic policy these factors are all in part making China 
less open to suggestions on economic policy from the US. It would be 
political suicide for a Chinese politician to appear as if he or she 
were taking orders from abroad.  Just like US politicians gain domestic 
support from appearing tough on China, Chinese politicians gain from 
having a tough stance on US politician meddling in Chinese affairs.

% The central government in Beijing, sets all macroeconomic policy, 
% including the exchange rate regime
% 
% 
% Politburo Standing Committee
% It is the central government in Beijing that sets all macroeconomic 
% policy, including the exchange rate regime.\footnote{\cite[pp. 
% 3]Naughton2008}.}
% 
% 
% 
% Exporting firms were concerned by the 2005-2008 appreciation, altough up 
% to the economic crisis in 2008 there was still strong growth in the 
% exporting sector.\footnote{\cite{Naughton2008}.}

%fmolo: I'm trying to separate Chinese diplomatic issues from US 
%diplomatic issues.
%but as Levy details in \cite{Levy11}, China has so far kept Obama from 
%living up to his promise, something that was ardently pointed out in the 
%election of 2012. China's motives behind avoiding the label `currency 
%manipulator' is closely connected with the us political debate where the 
%label could be an excuse for eager senators to mandate tariffs on 
%Chinese imports.
