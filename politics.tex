
\section{Politics}

The diplomatic debate between the US and China takes many different 
shapes. While the dialogue is defined by continuous attempts from the US 
to convince China to appreciate the RMB, the methods span from passive 
domestic policy statements to efforts of using international bodies to 
persuade China to change their economic policies.

In this section we will explore the different means of diplomatic 
pressure that has been in use during the last decade. For each method we 
will look at the following characteristics:

\begin{enumerate}
	\item{The Argumentation: What is the specific reasoning brought 
		forth in the economic debate?}
	\item{The Channels of Communication: Where does the debate take 
		place and with what authority?}
	\item{The response: What is the reaction to the debate from China's 
		side?}
	\item{The Context: Under what political or economic circumstances is 
		the debate brought forth?}
\end{enumerate}

% I would like a word or two here to round of the paragraph. Ideally 
% about how the characteristics help to understand what's going on in 
% the debate since it's important to understand that a critique from the 
% US side can be understood on many levels.

\subsection{Pressure Points}


\subsection{The Domestic Debate}
With each American election, a new batch of politicians parade their 
toughness on China's economic policies promising tougher measures and 
severe consequences. This is a popular agenda amongst many Americans who 
see the consequences of China's entrance on the global manufacturing 
market in their own lives. For every American product out competed by a 
cheaper Chinese alternative American jobs are lost, and a candidate who 
chastises China and vows to take measures to stop this will likely be 
received well.

This point hasn't escaped the players on the political scene who 
regularly promises to label China as a currency manipulator if they are 
elected\footnote{Recent examples include Barack Obama in 
2008\cite{Obama2008} and Mitt Romney in 2012\cite{Romney2012}}.  

Both candidates follow a remarkably similar argumentation to support 
their claim, basing their argument on the trade ramifications of a 
purportedly under appreciated RMB.  It is telling that Obama is making 
this statement during a speech to the `Alliance of American 
Manufacturing' (An interest organization representing a part of the US 
manufacturing industry).

Labeling China a `currency manipulator' doesn't do anything in itself, 
but as Levy details in \cite{Levy11}, China has so far kept Obama from 
living up to his promise, something that was ardently pointed out in the 
election of 2012. China's motives behind avoiding the label `currency 
manipulator' is closely connected with the us political debate where the 
label could be an excuse for eager senators to mandate tariffs on 
Chinese imports.

In 2009 a bill was proposed by the republican senator Timothy Ryan 
(Ohio), aiming to introduce a tariff with the stated purpose 
``\textit{To amend title VII of the Tariff Act of 1930 to clarify that 
	countervailing duties may be imposed to address subsidies relating 
	to a fundamentally undervalued currency of any foreign 
country}''\cite{Ryan2009}. The bill died in the senate after passing the 
house, but had it been instated it would make it very likely that China 
would had been forced to implement equal measures leading to a trade war 
with negative economic consequences for both nations as argued by Levy 
in chapter 20 of \cite{Evenett10}.

Official China is silent under these debates but it can be argued that 
the hostility towards China reflected by the tough stances are reflected 
by a similar attitude amongst Chinese politicians seeking to appear 
strong by not caving in to American demands\footnote{As presented in 
\cite{Levy11}}. The domestic debate on economic policy in China is 
focused much more on the continuation of growth than on the role of 
China economic policies in an international context. 
% TODO: some more on china should be added, but I'm having a minor 
% writers block here
% And interesting sources could be this article: 
% http://www.nytimes.com/2012/06/17/world/asia/in-shift-china-stifles-debate-on-economic-change.html?pagewanted=all&_r=0

\subsection{Behind Closed Doors}

The most direct way for the US administration to put pressure on China 
is for them to talk directly with Chinese policy makers, trying to 
influence their decisions. Since these conversations are rarely released 
for public consumption, there are very few indicators of the nature of 
the arguments. What can be examined however is the argumentation put 
forth by high standing officials presenting the US in their public 
speeches. It is of course not given that the bilateral US-Sino 
diplomatic exchanges follow the same arguments, but assessing the 
arguments that are laid forth by public figures might give us a gist of 
their contents.

One interesting example of such a public figure is the head of the US 
Treasure, Ben Bernanke. In a speech at the Chinese Academy of Social 
Sciences, Beijing he argued that: ``\textit{Greater scope for market 
	forces to determine the value of the RMB would also reduce an 
	important distortion in the Chinese economy, namely, the effective 
	distortion that an undervalued currency provides for Chinese firms 
	that focus on exporting rather than producing for the domestic 
market''}\footnote{The transcribed speech \cite{Bernanke06} used the 
term `subsidy' which created a lot of debate in the US, but the word was 
never uttered by Bernanke himself while giving the actual 
speech\cite{reuters06}}. Given to a Chinese audience, Bernanke attempted 
to make the point that Chinese consumers would benefit from a stronger 
RMB as it would give them access to cheaper imports and strengthen 
Chinese companies on the domestic market. With this rhetoric he mirrors 
the politicians focusing on the currency value as a driver of product 
prices.

However, in the same Speech Bernanke points out that one of the most 
effective ways to increase the welfare of Chinese households would be to 
reduce the domestic savings rate. This argument plays nicely together 
with a line of reasoning he presents at a lecture in Virginia a year 
earlier, which shows a significant shift away from the manufacturing 
driven argumentation\footnote{The speech can be read found here 
\cite{Bernanke05}}.

In this speech the main argument centers around how the savings rate 
of Chinese citizens distorts international economy. Here Bernanke argues 
that the poor welfare offered by the Chinese state forces a most Chinese 
to set aside a lot of money for retirement and medical self insurance.  
As the money is deposited in bank accounts, it gets reinvested in 
domestic and international projects, and particularly the US is a big 
receiver of foreign investment, which creates a current account deficit.  
The main point of this argument is the idea that the current account 
deficit or surplus is largely driven by the international economic 
environment and as such is not solely a domestic issue. Following this 
line of reason, it is not solely the responsibility of the US to 
eliminate their deficit with budget cuts, but equally the responsibility 
of large international players like to China to take measures to assure 
a balanced world economic. In the particular case of China this could be 
done for example by strengthening their currency.

These speeches were given at a point in time where China was giving up 
their peg of the RMB to the Dollar, slowly increasing the value over a 4 
year period between 2006 and 2010.

\subsection{International Bodies}

In the world of international finance there are several agencies 
responsible for coordinating various aspects of the world economy. A big 
part of the role of these organizations is to offer an avenue to settle 
disputes between member nations. % Jonas: Needs rewriting

The three most relevant of these organizations in this context is the 
G20, the International Monetary Fund (IMF) and the World Trade 
Organization (WTO).

The G20 is a group of 20 finance ministers and central bank governors 
from 20 major economies that try to promote economic cooperation and a 
venue for discussions on the international finance system







% We will also need to look at the distrust on both sides: In the US 
% politicians try to look strong by playing tough on China, and chinese 
% politicians act the same way.


% Include the remarks from Evenett chapter 7 arguing that the US 
% Treasury report's judgement of the chinese RMB statistically 
% correlates with domestic factors such as unemployment that shouldn't 
% play a role if the goal was to objectively asses the RMB dollar 
% Balance
