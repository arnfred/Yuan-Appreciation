\section{The political debate}
\label{sec:politics}

The following graph depicts the movement of the RMB real exchange rate against the US Dollar:

%TODO: Insert graph: real exchange rate from ca. 1994-2012

The RMB was pegged to the Dollar from 1994 at a rate of 8.28 RMB per 
Dollar until 2005. Then it appreciated at about 6\% per year, before the 
appreciation came to a halt in the aftermath of the financial crisis.  
The RMB appreciated again from mid 2010 up to now.

Following this movement of the RMB exchange rate, this chapter will depict the political debates about RMB appreciation in the USA and China.


\subsection{The political debate in the USA}

The history of the political debate in the USA about RMB undervaluation can roughly be seen in four phases:

It started shortly after China had joined the World Trade Organization WTO in 2001. As a member of the WTO, China gained access to Western markets (and vice versa) within an international regulatory structure ensuring free trade and banning protectionist measures. Critics in the USA argued that China was taking unfair advantage of these rules and bending them to China's own advantage. At the core of the critics of Chinese foreign exchange policy were the American labor unions.\footnote{\cite[pp. 14]{Levy2011}} Organized labor in the US has been critical of free trade in general, at least since the North American Free Trade Agreement (NAFTA) had been signed into law in 1993, and due to the ongoing decline of manufacturing in the US. In addition, the decline of products `made in USA'  was contrasted with ever more products labelled `made in China', as a result of China's export surge. Labor unions held the position that this decline was at least partly due to employers shifting production from the US to China. In 2004, the Assistant Director for International Economics at AFL-CIO, the umbrella federation for US unions, stated before the US Congress: 

\begin{quote}
[\dots]it is clear that the Chinese government’s manipulation of its currency, violation of international trade rules, and egregious repression of its citizens’ fundamental democratic and human rights are key contributors to an unfair competitive advantage.\footnote{AFL-CIO on U.S.-China Ties: Reassessing the Economic Relationship, http://www.dossiertibet.it/news/afl-cio-us-china-ties-reassessing-economic-relationship}
\end{quote}

Speaking of currency \emph{manipulation} and an \emph{unfair} competitive advantage, the AFL-CIO Director makes clear that she does not consider China's exchange rate policy as purely technical but as an issue where political action of the USA is needed to protect the interests of American manufacturing workers.

In the early 2000's, labor unions were largely alone in their call for 
action against China's currency policy. In a 2003 issue of \emph{The 
international economy magazine} over thirty international analysts from 
business and academia shortly answered the question `Is China's currency 
dangerously undervalued and a threat to the global 
economy?'.\footnote{\cite{IEM2003}.}
Most US analysts held the view, that even if China's currency was 
undervalued, export surplus did not come at the cost of US 
manufacturing. In their view, manufacturing was inevitably moving from 
the US to emerging economies. If the RMB exchange rate had been 
manipulated, it had been so at the cost of other emerging economies like 
India or Vietnam. Correspondingly, experts representing other East Asian 
countries in the \emph{International economy magazine} tended to a more 
harsh view of China's exchange rate policy. In addition, US experts 
pointed out the benefits of cheap Chinese exports for American 
consumers, who could choose from a growing variety of cheap consumer 
products like toys and electronics. Thus, cheap Chinese exports helped 
keeping US inflation low.

In 2005 the \emph{saving glut hypothesis} brought forward by Ben 
Bernanke opend a new line of criticism on China's export 
surplus.\footnote{as explained in section ...} %TODO: cite section
In this view, the Asian export surpluses posed a threat to global 
economic stability. However, since this critique came in a time of 
worldwide economic prospoerity, and since the RMB had started to 
appreciate against the US Dollar anyway, the critique was not initially 
drawn into the political spotlight.\footnote{\cite[p.16]{Levy2010}.}

During the \emph{Great Recession}, the financial and economic crisis that started with the burst of the US housing bubble in 2006, the critique of China intensified. As the view prevailed that the crisis was partly due to lack of demand and excess of savings worldwide, prominent economists like Paul Krugman - writing for the New York Times - joined the critics.\footnote{\cite{Krugman2009}.} In addition, the continuous appreciation of the RMB had stopped in 2008. As a result, the critique became more heated on a political level. With each American election, a new batch of politicians paraded their 
toughness on China's economic policies promising tougher measures and 
severe consequences. The most prominently displayed examples are of 
course American presidential candidates during election years.  Both 
Mitt Romney and Barack Obama made remarkably similar points during the 
elections of 2008 and 2012 respectively. At a speech in 2008 to the 
``Alliance of American Manufacturing'' an interest organization 
representing a part of the US manufacturing industry, Obama stated 
(speaking about China): ``Look, here's the bottom line. You guys keep on 
manipulating your currency and we are going start shutting off access to 
some of our markets''. Romney one the other hand stated several times 
during the 2012 election to label China as a currency manipulator from 
day one in office.\footnote{\cite{Obama2008} and 
\footnote{\cite{Romney2012}.}} These points are both reiterating an 
argument based on the trade ramifications of a purportedly undervalued 
RMB.

The debate in the US reached a climax in the spring of 2010 when the 
Obama administration had been long overdue to release the biannual 
treasury report. The report is mandated to label China as a currency 
manipulator if there is sufficient reason to do so, and considerable 
pressure was mounting from both politicians and pundits to do just that.  
After delaying the report several months, China was finally not 
officially accused of manipulating their currency by the report, a move 
that only increased the heated debate on the subject and made many 
republicans accuse Barack Obama of being weak on China.

\subsection{The political debate in China}


Political discussions of the RMB issue in China are of course much less public and 
transparent than the debate in the USA. It is the State Council in 
Beijing, headed by the Premier Minister - until 2012 this was Wen Jiabao 
- that sets macroeconomic policy, including the exchange rate regime.  
This policy ultimately has to be approved by the Politburo Standing 
Committee, headed by Hu Jintao until 
2012.\footnote{\cite{naughton2008}}. Instead of following the debate we 
will focus on different motivations and interest groups that might 
influence this political sphere.

The interest group most directly concerned by the exchange rate policy 
are Chinese exporting firms. Exporters generally have a strong interest 
in a weak RMB, since it makes their products cheaper and thus more 
competitive on the world market. The export industry is mainly located 
in eight provinces along the Chinese coast. Exporting firms make up only 
5-6\% of Chinese employment, but their influence is amplified by their 
enormous profits and their close relations to the government, especially 
through politicians of said provinces.\footnote{\cite[p.  
202]{Breslin2010}.} 

Also in favor of expansive monetary policy are many representatives of 
local Chinese government. Their main legitimization --- and sometimes 
their main source of income --- is economic growth.\footnote{\cite[pp.  
19]{Levy2011}.} Growth is also the only legitimization of the central 
government in Beijing, giving a strong rationale for growth-spurring 
monetary policy, including a low RMB exchange rate.

As mentioned in the section on economics, one problem in maintaining a 
low exchange rate is inflation. Inflation can be politically disruptive, 
spurring popular discontent. Accordingly, controlling inflation is one 
of the main macroeconomic goals of the central 
government.\footnote{\cite{Naugthon2011}.}

Officials concerned with inflation are joined by officials aiming at 
balancing the Chinese economy more towards domestic consumption, and 
away from the production surplus. Pushing for this are more traditional 
`communists' who feel Chinese workers did not profit enough from the 
economic boom, but `balancing the economy' has by now become an official 
part of economic policy.\footnote{\cite{Xinhua2011}.}

Chinese nationalism is a third factor that joins inflation control and 
growth as a motivation force. The perceived humiliations of the 19th 
century together with China's lackluster economic performance up until 
recent times has fueled a political will to reinstante China in its 
rightful place.\footnote{\cite{hughes2006}} This has lead to a political 
atmosphere where much weight is put on not appearing weak in 
confrontations with western powers.

In terms of economic policy these factors are all in part making China
less open to suggestions on economic policy from the US. The economic 
motivations lead China do disregard foreign concerns while the national 
pride makes it unlikely for a Chinese politician to appear as if he or 
she were taking orders from abroad.  

With these motives in mind we can now identify the two turning points, 
in 2008 and in 2010, of the RMB exchange rate. Before the year 2008 
China saw spectacular growth rates.  The main concerns then were 
preventing inflation and balancing the economy - the concerns of 
exporters were marginal and China let the RMB slowly appreciate against 
the US Dollar. In spring 2008 the global economic crisis hit exporters 
hard and growth was coming to a halt. At the same time a new economic 
team at the State Council was enacted in which according to Barry 
Naughton Wang Qishan, a official friendly to exporting firms, was the 
new strong man.\footnote{\cite{Naughton2008}.} This team decided a pause 
in RMB appreciation and pegged the RMB to the US Dollar at about 6.8 
Yuan per Dollar. In addition, the new policies included tax reliefs and 
easier access to loans for exporting firms. With this stimulus, the 
exporting sector was soon thriving again and by 2010 inflation again 
became the major concern.\footnote{\cite{Naughton2010}.} The peg was 
loosened and since 2010 the RMB is appreciating again. Although the 
appreciation is slower than between 2005 and 2008, the appreciation in 
real terms is even faster, due to Chinese inflation and \emph{de}flation 
in most Western countries.
