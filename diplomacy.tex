\section{Diplomatic options}


The diplomatic debate between the US and China takes many different 
shapes. While the dialogue is defined by continuous attempts from the US 
to convince China to appreciate the RMB, the methods span from passive 
domestic policy statements to efforts of using international bodies to 
persuade China to change their economic policies.

In this section we will explore the different means of diplomatic 
pressure that has been in use during the last decade. For each method we 
will look at the following characteristics:

\begin{enumerate}
	\item{The Argumentation: What is the specific reasoning brought 
		forth in the economic debate?}
	\item{The Channels of Communication: Where does the debate take 
		place and with what authority?}
	\item{The response: What is the reaction to the debate from China's 
		side?}
	\item{The Context: Under what political or economic circumstances is 
		the debate brought forth?}
\end{enumerate}

% I would like a word or two here to round of the paragraph. Ideally 
% about how the characteristics help to understand what's going on in 
% the debate since it's important to understand that a critique from the 
% US side can be understood on many levels.


>>>>>>> merged bibliography.bib with yuan.bib (a file I had initially overlooked)
>>>>>>> merged bibliography.bib with yuan.bib (a file I had initially overlooked)
\subsection{Unilateral action}

The most discussed and most direct option for the US would be to label China a `currency manipulator'. The US Treasury department is mandated in writing a biannual report on whether trading partners manipulate their currency.  

Labeling China a `currency manipulator' doesn't do anything in itself - it merely amounts to an international insult. But with labelling a country a currency manipulator the US Treasury Department has the mandate to
\begin{quote}
[\dots] take action to initiate negotiations with such foreign countries on an expedited basis, in the International Monetary Fund or bilaterally, for the purpose of ensuring that such countries regularly and promptly adjust the rate of exchange between their currencies and the United States dollar [\dots] to eliminate the unfair advantage.\footnote{OMNIBUS TRADE AND COMPETITIVENESS ACT OF 1988 (H.R. 3)
SEC. 3004. INTERNATIONAL NEGOTIATIONS ON EXCHANGE RATE AND ECONOMIC POLICIES., http://www.treasury.gov/resource-center/international/exchange-rate-policies/Documents/authorizing-statute.pdf}
\end{quote}

A



%fmolo: I'm trying to separate Chinese diplomatic issues from US diplomatic issues.
%but as Levy details in \cite{Levy11}, China has so far kept Obama from 
%living up to his promise, something that was ardently pointed out in the 
%election of 2012. China's motives behind avoiding the label `currency 
%manipulator' is closely connected with the us political debate where the 
%label could be an excuse for eager senators to mandate tariffs on 
%Chinese imports.

In 2009 a bill was proposed by the republican senator Timothy Ryan 
(Ohio), aiming to introduce a tariff with the stated purpose 
``\textit{To amend title VII of the Tariff Act of 1930 to clarify that 
	countervailing duties may be imposed to address subsidies relating 
	to a fundamentally undervalued currency of any foreign 
country}''\cite{Ryan2009}. The bill died in the senate after passing the 
house.

% but had it been instated it would make it very likely that China 
%would had been forced to implement equal measures leading to a trade war 
%with negative economic consequences for both nations as argued by Levy 
%in chapter 20 of \cite{Evenett10}.

%Official China is silent under these debates but it can be argued that 
%the hostility towards China reflected by the tough stances are reflected 
%by a similar attitude amongst Chinese politicians seeking to appear 
%strong by not caving in to American demands\footnote{As presented in 
%\cite{Levy11}}. The domestic debate on economic policy in China is 
%focused much more on the continuation of growth than on the role of 
%China economic policies in an international context. 


% TODO: some more on china should be added, but I'm having a minor 
% writers block here
% And interesting sources could be this article: 
% http://www.nytimes.com/2012/06/17/world/asia/in-shift-china-stifles-debate-on-economic-change.html?pagewanted=all&_r=0

\subsection{To argue convincingly}

%fmolo: I fear we don't have a lot to say about closed-door-negotiations. As I got Evenett he doesn't think it is important since the US have no direct leverage. If anything, they could exert pressure through multilateral efforts. I therefore put the following paragraphs in another context

%The most direct way for the US administration to put pressure on China 
%is for them to talk directly with Chinese policy makers, trying to 
%influence their decisions. Since these conversations are rarely released 
%for public consumption, there are very few indicators of the nature of 
%the arguments. What can be examined however is the argumentation put 
%forth by high standing officials presenting the US in their public 
%speeches. It is of course not given that the bilateral US-Sino 
%diplomatic exchanges follow the same arguments, but assessing the 
%arguments that are laid forth by public figures might give us a gist of 
%their contents.

%TODO: Find (remember) sources/quotations for this:
Critics of China sometimes supplement their case with the assertion, that a higher-valued RMB would actually be in the interest oft the Chinese people, if not its export sector.

For an ordinary Chinese factory worker, the cheap RMB might on one hand 
mean that their factory is doing great on foreign markets, but at the 
same time buying foreign products like European cars or American gadgets 
becomes very expensive. 

Additionally the artificially low interest rates, in combination with 
inflation, deprive Chinese citizens of attractive saving options on 
their bank accounts. Due to capital controls they cannot place their 
money in foreign banks where interest rates are higher. In absence of a 
comprehensive social welfare state, retirement provisions become a major 
concern for many citizens.

Lacking other saving opportunities, many Chinese invest their money in 
real estate as a saving asset, driving real estate prices up.  This 
confronts less affluent citizens with serious difficulties when they are 
looking for habitation in urban areas.  Infamously high housing prices, 
e.g.  in Shanghai, have become a major social issue, also reflected in 
popular culture. 
%TODO: Cite movies (one of them seen in the Summer 
%School), otherwise scrap the last part.

One interesting example of an important public figure arguing alon these lines is the head of the US 
Federal Reserve, Ben Bernanke. In a speech at the Chinese Academy of Social 
Sciences in Beijing he argued that: ``\textit{Greater scope for market 
	forces to determine the value of the RMB would also reduce an 
	important distortion in the Chinese economy, namely, the effective 
	distortion that an undervalued currency provides for Chinese firms 
	that focus on exporting rather than producing for the domestic 
market''}\footnote{The transcribed speech \cite{Bernanke06} used the 
term `subsidy' which created a lot of debate in the US, but the word was 
never uttered by Bernanke himself while giving the actual 
speech\cite{reuters06}}. Given to a Chinese audience, Bernanke attempted 
to make the point that Chinese consumers would benefit from a stronger 
RMB as it would give them access to cheaper imports and strengthen 
Chinese companies on the domestic market. With this rhetoric he mirrors 
the politicians focusing on the currency value as a driver of product 
prices.

However, in the same Speech Bernanke points out that one of the most 
effective ways to increase the welfare of Chinese households would be to 
reduce the domestic savings rate. This argument plays nicely together 
with a line of reasoning he presents at a lecture in Virginia a year 
earlier, which shows a significant shift away from the manufacturing 
driven argumentation\footnote{The speech can be read found here 
\cite{Bernanke05}}.

In this speech the main argument centers around how the savings rate 
of Chinese citizens distorts international economy. Here Bernanke argues 
that the poor welfare offered by the Chinese state forces a most Chinese 
to set aside a lot of money for retirement and medical self insurance.  
As the money is deposited in bank accounts, it gets reinvested in 
domestic and international projects, and particularly the US is a big 
receiver of foreign investment, which creates a current account deficit.  
The main point of this argument is the idea that the current account 
deficit or surplus is largely driven by the international economic 
environment and as such is not solely a domestic issue. Following this 
line of reason, it is not solely the responsibility of the US to 
eliminate their deficit with budget cuts, but equally the responsibility 
of large international players like to China to take measures to assure 
a balanced world economic. In the particular case of China this could be 
done for example by strengthening their currency.

These speeches were given at a point in time where China was giving up 
their peg of the RMB to the Dollar, slowly increasing the value over a 4 
year period between 2006 and 2010.

\subsection{Multilateral action}

In the world of international finance there are several agencies 
responsible for coordinating various aspects of the world economy. A big 
part of the role of these organizations is to offer an avenue to settle 
disputes between member nations. % Jonas: Needs rewriting

The three most relevant of these organizations in this context is the 
G20, the International Monetary Fund (IMF) and the World Trade 
Organization (WTO).

The G20 is a group of 20 finance ministers and central bank governors 
from 20 major economies that try to promote economic cooperation and a 
venue for discussions on the international finance system. Pushed by the 
US several attempts at making the participating leaders pledge their 
allegiance to economic policies that would prevent countries from 
manipulating their currencies. However the strongest statement so far 
was reached at the 2009 Summit in Pittsburgh where the leaders agreed 
``\textit{to adopt the policies needed t lay the foundation for strong, 
sustained and balanced growth in the 21st century}''. Later attempts at 
further defining exactly what is covered by ``balanced'' have 
failed\footnote{An interesting account of this process can be found in 
\cite{Levy11}}.

An alternative approach would be to go through the IMF which stated goal 
is to stabilize exchange rates and lend money to countries that might 
need them. In fact Dominic Strauss Kahn has on behalf of the IMF 
publicly stated that the RMB is undervalued in 2009, and the IMF has 
repeated this sentiment with slight modifications since 
then\cite{reuters09}.  Despite these statements no action has been taken 
by IMF to regulate or penalize China, largely because the organization 
lacks the means to influence countries that aren't dependent on it for 
borrowing money.

Last the World Trade Organization is deliberately set up to supervise 
and liberalize international trade. In \cite{Evenett10} several chapters 
are devoted to an extended discussion about the interpretation of the 
ruleset laid out by WTO to ensure free trade. The central question being 
whether there are provisions in WTO agreements that can be used to 
prosecute China.  However as discussed in \cite{Levy11} the WTO is 
unable to cope with such a situation even if there were an article 
clearly stating that China was being guilty of economic misconduct.

In China the IMF labeling the RMB as undervalued elicited a strong 
response. After the label of the RMB as ``moderately undervalued'' in 
2011, the chief of the Institute of International Finance at the East 
China Normal University, Huang Zeming conceded that ``\textit{ it is 
	certain that the RMB exchange rate has been underrated, but there is 
	no fixed standard that could be used to test whether the appraisal 
method is proper}''\footnote{A more in depth response can be found in 
\cite{ChinaDaily11}}, arguing that the conclusions made by the IMF are 
meaningless.





% We will also need to look at the distrust on both sides: In the US 
% politicians try to look strong by playing tough on China, and chinese 
% politicians act the same way.


% Include the remarks from Evenett chapter 7 arguing that the US 
% Treasury report's judgement of the chinese RMB statistically 
% correlates with domestic factors such as unemployment that shouldn't 
% play a role if the goal was to objectively asses the RMB dollar 
% Balance
