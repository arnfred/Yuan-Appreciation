\section{Diplomatic options}
\label{sec:diplomacy}


The diplomatic debate between the US and China takes many different 
shapes. While the dialogue is defined by continuous attempts from the US 
to convince China to appreciate the RMB, the methods span from passive 
domestic policy statements to efforts of using international bodies to 
persuade China to change their economic policies.

In this section we will explore the different means of diplomatic 
pressure that has been in use during the last decade. For each method we 
will look at the following characteristics:

\begin{enumerate}
	\item{The Argumentation: What is the specific reasoning brought 
		forth in the economic debate?}
	\item{The Channels of Communication: Where does the debate take 
		place and with what authority?}
	\item{The response: What is the reaction to the debate from China's 
		side?}
	\item{The Context: Under what political or economic circumstances is 
		the debate brought forth?}
\end{enumerate}

% I would like a word or two here to round of the paragraph. Ideally 
% about how the characteristics help to understand what's going on in 
% the debate since it's important to understand that a critique from the 
% US side can be understood on many levels.


\subsection{Unilateral action}

The most discussed and most direct option for the US would be to label China a `currency manipulator'. The US Treasury department is mandated in writing a biannual report on whether trading partners manipulate their currency.  

Labeling China a `currency manipulator' doesn't do anything in itself - it merely amounts to an international insult. But with labelling a country a currency manipulator the US Treasury Department has the mandate to
\begin{quote}
[\dots] take action to initiate negotiations with such foreign countries on an expedited basis, in the International Monetary Fund or bilaterally, for the purpose of ensuring that such countries regularly and promptly adjust the rate of exchange between their currencies and the United States dollar [\dots] to eliminate the unfair advantage.\footnote{OMNIBUS TRADE AND COMPETITIVENESS ACT OF 1988 (H.R. 3)
SEC. 3004. INTERNATIONAL NEGOTIATIONS ON EXCHANGE RATE AND ECONOMIC POLICIES., http://www.treasury.gov/resource-center/international/exchange-rate-policies/Documents/authorizing-statute.pdf}
\end{quote}


%fmolo: I'm trying to separate Chinese diplomatic issues from US diplomatic issues.
%but as Levy details in \cite{Levy11}, China has so far kept Obama from 
%living up to his promise, something that was ardently pointed out in the 
%election of 2012. China's motives behind avoiding the label `currency 
%manipulator' is closely connected with the us political debate where the 
%label could be an excuse for eager senators to mandate tariffs on 
%Chinese imports.

So far, the US Treasury Department refrained from labelling China a currency manipulator, much to the dismay of members of US congress, especially those representing states with strong manufacturing.
In 2009 a bill was proposed by the republican senator Timothy Ryan 
of Ohio, aiming to introduce a tariff with the stated purpose 
``\textit{To amend title VII of the Tariff Act of 1930 to clarify that 
	countervailing duties may be imposed to address subsidies relating 
	to a fundamentally undervalued currency of any foreign 
country}''\cite{Ryan2009}. Because the low RMB has the same effect as a tariff on US goods, the reasoning goes, the US should reply with the same means, imposing a tariff on Chinese goods. The bill passed the 
House but was not put to a vote in the Senate, sharing the fate with other similar bills proposed in Congress. 

The USA have in fact tried the proposed method before: In 1971 Nixon decided to instate trade 
tariffs against goods imported from Japan and Germany after having 
pressured them into appreciating their currency in vain. The tariffs 
forced Japan to appreciate the Yen against the Dollar shortly after, but 
the overall economic gains for the US are debatable.\footnote{\cite{kuroda2004}.} 

In the actual case, China would very probably have reacted to US tariffs by implementing equal measures, 
leading to a trade war with negative economic consequences for both 
nations.\footnote{\cite{Levy2010}.}

Moreover, US tariffs would clearly be in violation of WTO regulations of free trade. As a consequence, the USA would have to find allies among WTO members to support their position. In other words, the US would have to take \emph{multilateral} diplomatic efforts.

% but had it been instated it would make it very likely that China 
%would had been forced to implement equal measures leading to a trade war 
%with negative economic consequences for both nations as argued by Levy 
%in chapter 20 of \cite{Evenett10}.

%Official China is silent under these debates but it can be argued that 
%the hostility towards China reflected by the tough stances are reflected 
%by a similar attitude amongst Chinese politicians seeking to appear 
%strong by not caving in to American demands\footnote{As presented in 
%\cite{Levy11}}. The domestic debate on economic policy in China is 
%focused much more on the continuation of growth than on the role of 
%China economic policies in an international context. 

% TODO: some more on china should be added, but I'm having a minor 
% writers block here
% And interesting sources could be this article: 
% http://www.nytimes.com/2012/06/17/world/asia/in-shift-china-stifles-debate-on-economic-change.html?pagewanted=all&_r=0

\subsection{Multilateral action}

There are three international bodies that govern multilateral economic relations: The 
G20, the International Monetary Fund (IMF) and the World Trade 
Organization (WTO).

The G20 is a group of 20 finance ministers and central bank governors 
from 20 major economies that try to promote economic cooperation and a 
venue for discussions on the international finance system. Pushed by the 
USA several attempts at making the participating leaders pledge their 
allegiance to economic policies that would prevent countries from 
manipulating their currencies. However the strongest statement so far 
was reached at the 2009 Summit in Pittsburgh where the leaders agreed 
`\textit{to adopt the policies needed to lay the foundation for strong, 
sustained and balanced growth in the 21st century}'. Later attempts at 
further defining exactly what is covered by `balanced' have 
failed.\footnote{An interesting account of this process can be found in 
\cite{Levy11}.}

An alternative approach would be to address the IMF, whose stated goal it
is to stabilize exchange rates. In fact Dominic Strauss Kahn, then head of the IMF, has 
publicly stated that the RMB is undervalued in 2009, and the IMF has 
repeated this sentiment with slight modifications since 
then.\footnote{\cite{reuters09}.}  Despite these statements no action has been taken 
by the IMF to regulate or penalize China, largely because the organization 
lacks the means to influence countries that aren't dependent on it for 
borrowing money.

Last the World Trade Organization is deliberately set up to supervise 
and liberalize international trade. In \cite{Evenett10} several chapters 
are devoted to an extended discussion about the interpretation of the 
ruleset laid out by WTO to ensure free trade. The central question being 
whether there are provisions in WTO agreements that can be used to 
prosecute China.  However as discussed in \cite{Levy11} the WTO is 
unable to cope with such a situation even if there were an article 
clearly stating that China was being guilty of economic misconduct.

In China the IMF labeling the RMB as undervalued elicited a strong 
response. After the label of the RMB as ``moderately undervalued'' in 
2011, the chief of the Institute of International Finance at the East 
China Normal University, Huang Zeming conceded that ``\textit{ it is 
	certain that the RMB exchange rate has been underrated, but there is 
	no fixed standard that could be used to test whether the appraisal 
method is proper}''\footnote{A more in depth response can be found in 
\cite{ChinaDaily11}}, arguing that the conclusions made by the IMF are 
meaningless.

\subsection{The hypothetical case}

To illuminate the diplomatic dead end the US administration has found 
themselves in, it is illustrative to explore a few scenarios of what 
would have happened had the US decided to take a step further.

In order to persuade China to do anything, the US must be able to either 
coerce China into doing so, or be able to offer something in return.  
Coercion can happen either through military force or economic pressure 
while agreements can be reached only when both parts think they gain a 
benefit from doing so.

In terms of international trade it could be argued that it is mutually 
beneficial for the world economy when the current account surplus or 
deficit of every nation is kept within a certain factor of their 
GDP\footnote{This argument is put forward by for example 
\cite{cline2009}}. While the US has tried to influence the G20 by having 
them include the word `balanced' in a central statement about the global 
economy further efforts have not been fruitful. However had this been a 
high priority for the US, it seems likely that they could have exerted 
diplomatic pressure on other nations to commit to a clear defined policy 
by for example making it easier for foreign nationals to invest on the 
US market, or threatening to make it more difficult if other countries 
were unwilling to sign.

The consequence of such a treaty if it were strictly worded would be 
that participating countries could face penalties in case they were 
running a budget deficit or trade balance surplus.  This is a huge dent 
in the right of most countries to autonomously decide on their monetary 
policy, and seeing that the US has been running a large current account 
deficit for most of the last decade, it's not clear that such a treaty 
would even be in the interest of the US.

Military coercion against China has most probably not been on the table.  
While it would be an effective way of forcing China to changing their 
economic policy, it would also destabilize the entire region and ruin 
the diplomatic relationship between the two countries. The US does have 
a large strategic presence in the area and it wouldn't be impossible to 
imagine that an agreement could have been reached where China would 
adjust their economic policy against the US exercising their influence 
with regards to specific issues with e.g. North Korea or Japan. This 
scenario is difficult to substantiate or for that manner disprove since 
in the realms of all possible worlds it could well have happened and 
could just as well be impossible for reasons that would never be known.  
Still, in light of the self-sufficient trait of Chinese policy it is 
probably unlikely that China would want to depend on US influence in 
these matters.



While these scenarios might not encompass all possible avenues of 
response from the US' side, it does illustrate how the most obvious 
actions that could be taken aren't currently beneficial for the US.

% Possible Scenarios in response to US pressure based on Chinese 
% Priorities (Non-interference, Economic Performance, Not appearing 
% weak).Conclude with how this is not beneficial for the US.

\subsection{To argue convincingly}

Critics of China sometimes supplement their case with the assertion, 
that a higher-valued RMB would actually be in the interest of the 
Chinese people, if not its export sector.

For an ordinary Chinese factory worker, the cheap RMB might on one hand 
mean that their factory is doing great on foreign markets, but at the 
same time buying foreign products like European cars or American gadgets 
becomes very expensive. 

Additionally the artificially low interest rates, in combination with 
inflation, deprive Chinese citizens of attractive saving options on 
their bank accounts. Due to capital controls they cannot place their 
money in foreign banks where interest rates are higher. In absence of a 
comprehensive social welfare state, retirement provisions become a major 
concern for many citizens.

Lacking other saving opportunities, many Chinese invest their money in 
real estate as a saving asset, driving real estate prices up.  This 
confronts less affluent citizens with serious difficulties when they are 
looking for habitation in urban areas.  Infamously high housing prices, 
e.g.  in Shanghai, have become a major social issue, also reflected in 
popular culture. 

One interesting example of an important public figure arguing along 
these lines is the head of the US Federal Reserve, Ben Bernanke whom 
we've mentioned earlier. With his theory of the `savings glut' he has 
been trying to persuade China that their economic policies are not only 
hurting the international markets but also turns out to have a negative 
effect on most Chinese citizens.  

In a speech at the Chinese Academy of Social Sciences in Beijing he 
argued that: ``\textit{Greater scope for market forces to determine the 
	value of the RMB would also reduce an important distortion in the 
	Chinese economy, namely, the effective distortion that an 
undervalued currency provides for Chinese firms that focus on exporting 
rather than producing for the domestic market''}\footnote{The 
	transcribed speech \cite{Bernanke06} used the term `subsidy' which 
created a lot of debate in the US, but the word was never uttered by 
Bernanke himself while giving the actual speech\cite{reuters06}}.

In line with his `savings glut' theory, Bernanke went on to point out 
that one of the most effective ways to increase the welfare of Chinese 
households would be to reduce the domestic savings rate. This argument 
plays nicely together with a line of reasoning he presents at a lecture 
in Virginia a year earlier, which shows a significant shift away from 
the manufacturing driven argumentation\footnote{The speech can be read 
found here \cite{Bernanke05}}. In this speech the main argument centers 
around how the `savings glut' distorts international economy.  As the 
Chinese store their wealth in banks instead of spending it in the 
domestic market, it gets reinvested for a big part in international 
projects, and particularly the US is a big receiver of foreign 
investment, which in turn creates a current account deficit.  The main 
point of this argument is the idea that the current account deficit or 
surplus is largely driven by the international economic environment and 
as such is not solely a domestic issue. Following this line of reason, 
it is not solely the responsibility of the US to eliminate their deficit 
with budget cuts, but equally the responsibility of large international 
players like to China to take measures to assure a balanced world 
economic. In the particular case of China this could be done for example 
by strengthening their currency.

However while this argumentation was directed at China, there is no 
indication that it changed any policies. It is unlikely that Chinese 
economists were not already aware of at least some of the points that 
Bernanke proposed, and even if they weren't, there would be little 
political will to take advice from the US.
